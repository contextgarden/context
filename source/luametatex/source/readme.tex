% language=uk

\usemodule[abbreviations-logos]
\usemodule[article-basic]

\setuplayout[article][header=0pt]

\starttext

\starttitle[title={Welcome}]

This is a follow up on the \LUATEX\ project by Hartmut Henkel, Taco Hoekwater,
Hans Hagen en Luigi Scarso and friends, a project related to the \CONTEXT\ macro
package. The \LUATEX\ functionality became stable around version 1.10 and because
the engine is also used outside \CONTEXT, a follow up happens in another
namespace. This version is a stripped down variant and is mostly meant for
\CONTEXT. Some interfaces have been adapted a bit and the expectation is that we
will polish things more.

The source code is part of the \CONTEXT\ distribution and compilation is driven
by \type {cmake} instead of \type {autotools}. By keeping the code with \CONTEXT\
code, consistency is guaranteed: one can always generate the binary that relates
to the functionality expected. There are no dependencies on other code: all is
self contained.

The work name of this follow up is \LUAMETATEX\ which can be seen as \LUATEX\ 2.0
or higher. Of course \CONTEXT\ runs on top of \LUATEX, but a variant, tagged
\LMTX\ runs on \LUAMETATEX. One of the main ideas behind this project is that it
guarantees the integrity of \CONTEXT\ and the used engine and that we stick to
the principles of a lean and mean engine.

If you install a new binary for \CONTEXT\ the following is the intended use:

\starttyping
tex/texmf-platform/bin/luametatex[.exe]
tex/texmf-platform/bin/mtxrun           -> luametatex[.exe]
tex/texmf-platform/bin/context          -> luametatex[.exe]
tex/texmf-platform/bin/mtxrun.lua
tex/texmf-platform/bin/context.lua
\stoptyping

The binary is rather small so having a few copies (or links) is no problem. The
\type {mtxrun} and \type {context} stubs will launch \LUAMETATEX. No extra
programs are needed.

The files in this source tree cannot be dropped into the \LUATEX\ source tree:
they are different in many aspects. Although much has been done the codebase will
be stepwise cleaned up further and more documentation will be added. Background
information on how we came to this can be found in the \CONTEXT\ distribution,
for instance in the \type {followingup.pdf} document.

In addition to the names mentioned above I want to stress that other \CONTEXT\
developers are involved. For instance Mojca Miklavec manages compilation on the
build farm and deals with the installer at the \type {contextgarden}. Without
Alan Braslau and Wolfgang Schuster as conceptual sparing partners there would be
no \LUAMETATEX. Torture testing by users like Thomas Schmitz and Aditya Mahajan
who mix \TEX, \XML, \LUA, \PDF, and other functionality is instrumental. I can
mention more names, but it must be clear that what keeps me going in doing this
comes from the \CONTEXT\ community.

The work is far from finished. It's a stepwise process of going lean and mean,
reshuffling code, checking things out. Take into account that we want to remain
compatible as much as possible with stock \TEX, which includes the original
documentation (and therefore naming of variables). It will always be a mix of
\quotation {What we started with.}, \type {How it became.} and \type {How it
should be now.}: a rather hybrid evolution of one of the oldest public programs
out there.

As mentioned the code is part of the \CONTEXT\ distribution. We'll try to prevent
pollution and bloating of the code base as much as possible, also because that
way we get independent snapshots.

\blank[2*big,samepage]

\startlines
Hans Hagen
Pragma ADE
j.hagen @ xs4all . nl
\stoplines

\stoptitle

\stoptext
