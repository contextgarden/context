% language=us runpath=texruns:manuals/luametatex

\environment luametatex-style

\startdocument[title=Introduction]

The \LUAMETATEX\ manual that is a variant of the \LUATEX\ manual provides an
overview similar to its parent. Instead of adding more and more to that one, an
alternative take is provided. Here we start less form a historic perspective and
treat the engine as independent development. The main reason for this is that we
want to focus on \CONTEXT, if only because that is the macro package that uses it
and also drives the development.

In \LUAMETATEX\ we go further than in \LUATEX. We extend the language, refactor
most subsystems and assume that the macro package adapts to that. Of course we
are compatible as much as possible with predecessors but we also take the freedom
to tune some default behavior. For instance, moving on with math rendering means
that we can make assumptions with respect to fonts and because the math fonts
have issues that never will be solved we assume that the macro package is not
only to feed the engine with tweaked fonts that can use the engine to its maximum
extend. The same is true for more mechanism, like for instance the par builder,
% where we introduce multiple paragraph line break passes using features not
present in other engines. Although extensions like these are not discussed here we
do have to describe the underlying mechanisms and interfaces and thereby assume
usage as in \CONTEXT.

A manual like this evolves over time and will take years to complete. These are
volunteer efforts unless some project makes it possible to spend more time on it.
In practice most work on \TEX\ development is unpaid for and therefore mostly
driven by the joy of playing with typesetting and coming up with solutions for
problems that users present us. Keep that in mind when reading and wondering why
the focus is not on what you expect or what is best for marketing.

This manual replaces the older \LUAMETATEX\ manual. It has some less and some
more than its predecessor which was derived from the \LUATEX\ manual. It will
take some time to \quote {complete}. Eventually I might add a few registers but
it makes only sense when the manual is more stable and I have to be in the mood
to spend time on it.

\start \switchtobodyfont[small] \setupinterlinespace

Disclaimer. I don't use \quote {artificial intelligence} tools for development
and have no plans to do that either. If I can't manage without, I should not go
on with developments anyway. I don't want to use tools that rip-off code (and
basically abuse whatever people put on the internet for others to enjoy), pretty
much aim at control and advertising (its all about money), infringe copyright,
depend on other peoples originality and efforts, and frankly spoken, bring very
little to my table, while consuming extreme amount of energy world-wide. I've
nothing against expert systems applied wisely but that's a different story than
today's big tech, commerce and dominance driven AI fashion. I also don't jump on
every new language bandwagon because in the end there is little to gain, and all
these software religious claims don't impress in the end. It's a waste of time
and energy. Typesetting is very much also a human thing: look and feel,
perception, joy and human interaction. I like to see what (challenges) users come
up with, in results and demands; that is what drives me. Therefore, an important
main point here is that all errors and hallucinations in this manual are mine.

\par \stop

\starttabulate[|||]
    \NC Author      \NC Hans Hagen & friends        \NC \NR
    \NC \CONTEXT    \NC \contextversion             \NC \NR
    \NC \LUAMETATEX \NC \luametatexverboseversion   % \texengineversion
                        \space (dev id: \luametatexfunctionality)  \NC \NR
    \NC Support     \NC contextgarden.net & tug.org \NC \NR
\stoptabulate

\stopdocument
