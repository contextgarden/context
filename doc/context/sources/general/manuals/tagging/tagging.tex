% language=us runpath=texruns:manuals/tagging

% todo: use concrete

\usemodule
  [abbreviations-logos,scite,math-verbatim]

\setupbackend[format=pdf/ua-2]
\setuptagging[state=start]
% \nopdfcompression

\setupbodyfont
  [pagella,10pt]

\setupwhitespace
  [big]

\setuphead
  [chapter]
  [style=\bfc,
   interaction=all]

\setuphead
  [section]
  [style=\bfb]

\setuphead
  [subsection]
  [style=\bfa]

\setuphead
  [subsubsection]
  [style=\bf,
   after=]

\setuplist
  [interaction=all]

\setupdocument
  [before=\directsetup{document:titlepage}]

% The tag shape will be improved.

\startuseMPgraphic{titlepage}
    numeric n ;

    path p ; p :=
            (1,0)
        --- (5,0)
        ... (6,3)
        ... (4,4)
        --- (4,6)
        ... (3,7)
        ... (2,6)
        --- (2,4)
        ... (0,3)
        ... cycle
    ;

    for i=1 upto 21 :
        for j=1 upto 30 :
            draw image (
                fill
                    p
                    withcolor .7yellow ;
                fill
                    fullcircle shifted (3,6)
                    withcolor white ;
                n := -1 randomized 2 ;
                draw textext (
                    if n > 0.5 :
                        "\ttbf no\hskip.2em tag"
                    elseif n > 0 :
                        "\ttbf retag"
                    elseif n > -0.5 :
                        "\ttbf untag"
                    else :
                        "\ttbf tag"
                    fi
                ) ysized 3/2 shifted (3,3/2)
                    withcolor white ;
            ) rotated (-15 randomized 30)
                shifted (10i,10j) ;
        endfor ;
    endfor ;

    setbounds currentpicture to boundingbox currentpicture enlarged 3 ;

    addbackground withcolor .8blue ;

    currentpicture := currentpicture xysized(PaperWidth,PaperHeight) ;

    picture q[] ;

    q[1] := image (
        draw image (
            for i=1 upto 15 :
                for j=1 upto 14 :
                    fill
                        (fullsquare xyscaled (2,3))
                        shifted (3*i,4*j)
                    ;
                endfor ;
            endfor ;
        ) withcolor .7red ;
        draw image (
            for i=1 upto 15 :
                for j=1 upto 14 :
                    draw
                        textext("\ttbf MCID") rotated 90 xsized .8
                        shifted (3*i,4*j)
                    ;
                endfor ;
            endfor ;
        ) withcolor white ;
    )
      xsized .9PaperWidth
    ;

    q[2] := image (
        fill
            p
            withcolor .7green ;
        fill
            fullcircle shifted (3,6)
            withcolor white ;
        n := -1 randomized 2 ;
        draw textext ("\ttbf PDF")
            ysized 3/2 shifted (3,3/2)
            withcolor white ;
    )
      rotated -5
      xsized .25PaperWidth
    ;

    q[3] := image (
        fill
            p
            withcolor .7green ;
        fill
            fullcircle shifted (3,6)
            withcolor white ;
        n := -1 randomized 2 ;
        draw textext ("\ttbf tagged")
            ysized 3/2 shifted (3,3/2)
            withcolor white ;
    )
      rotated 5
      xsized .25PaperWidth
    ;

    q[1] := q[1]
        shifted -center topboundary q[1]
        shifted center topboundary currentpicture
        shifted (0,-PaperHeight/20)
    ;

    q[2] := q[2]
        shifted -center topboundary q[2]
        shifted center bottomboundary q[1]
        shifted (6.5PaperWidth/20,1.5PaperHeight/20)
    ;

    q[3] := q[3]
        shifted -center topboundary q[3]
        shifted center bottomboundary q[1]
        shifted (PaperWidth/20,1PaperHeight/20)
    ;

    draw q[1] withtransparency (1,.70) ;
    draw q[2] withtransparency (1,.85) ;
    draw q[3] withtransparency (1,.85) ;

\stopuseMPgraphic

\startsetups document:titlepage
    \startTEXpage
        \useMPgraphic{titlepage}
    \stopTEXpage
\stopsetups

\setuptyping[option=TEX]

\startdocument[title=foo]

\startchapter[title=Why do we tag]

Around 2010 tagged \PDF\ showed up in \CONTEXT. Apart from demonstrating that it
could be done it served little purpose because only full Acrobat could show a
structure tree and in the more than a decade afterwards no other viewer did
something with it. However for some users it was a necessity.

In 2024 we picked up on tagging because due to regulations (especially in higher
education) demands for tagged \PDF\ in the perspective of accessibility popped
up. We will not go into details here but just mention that we want to make sure
that users can meet these demands.

As of now (2024) we have little expectations when it comes to tagging. The
ongoing discussions about how to tag, how to interpret the specification, what to
validate, and what to expect from applications are likely to go on for a while,
so the best we can do is keep an eye on it and adapt when needed. If we have
opinions, these will be exposed in other documents (and articles).

{\em This manual is work in progress!}

\startlines
Hans Hagen
Mikael Sundqvist
\stoplines

\stopchapter

\startchapter[title=Tagging text]

As mentioned in the introduction, we need to satisfy validators that are imposed
on those working in education (often via web interfaces with little information
on what actually gets checked, it's business after all). It is not that hard to
fool them and make documents compliant, so that is what we can do anyway. It is
also possible to let these tools do some auto tagging but our experiments showed
that this is a disaster. So, we end up with a mix of relatively rich tagging that
we feel good with. When we're a decade down the road we expect that with a little
help from large language models a decent verbose tagging is better than a crappy
suboptimal one.

One reason for tagging is that it could permit extraction but there are better
solutions to that: if there is something shown in a table or graphic, why not add
the dataset. We currently add \MATHML\ and \BIBTEX\ blobs but more can become
possible in the future (this also depends on user demand).

Another application is reflow but when that is needed, why not go \HTML\ or
distribute different output. When accessibility is the target one has to wait
till more is clear how that is actually supposed to work. Often the
recommendations are to use Arial, little color, simple sectioning etc, so that
gives little reason to use \PDF\ at all.

All that said, we assume that \PDF\ level 2 is used, if only because it looks
like validators aim for that. Also, if you find pre level 2 documents produced
elsewhere, often tagging is so bad or weird that one can as well ignore it.

Tagging in a document is enabled with:

\starttyping
\setupbackend[format=pdf/ua-2]
\setuptagging[state=start]
\stoptyping

The first command ensures that the right data ends up in the \PDF\ file, and the
second one enables tagging. As long as you're working on a document you can
comment these commands which saves you some runtime and give way smaller files.
\footnote {With \type {\enabledirectives [backend.usetags=crap]} you can map to
old built-in ua-1 tag names but that will fail with version 2 validation because
it has a simplified (or restricted) model.}

\stopchapter

\startchapter[title=Tagging math]

Tagging math at level 2 is still experimental but works as follows. Instead of
tagging the atoms and structures, as we do in level 1, we generate a \MATHML\
attachment and put a so called actual text on the math structure node. This text
can be spoken by reading machinery. The \MATHML\ is not that rich but we can enable
more detail when needed. However, given the way (presentational) \MATHML\ evolved
we are somewhat pessimistic. Instead of adding a few more elements that would
help to provide structure, some features are dropped. Also, support in browsers
comes and goes, either native or depending on \JAVASCRIPT.

Because there is much freedom in how mathematical symbols and constructs are
used, you might need to help math tagging bit. The process is driven by group
sets that refer to domains. An example of a domain is chemistry. For now we just
mention that this features is there and as time flies by we can expect more
granular usage.

\starttyping
\definemathgroupset
  [mydomain]
  [every] % a list of dictionaries

\setmathgroupset
  [mydomain]
\stoptyping

For now you can ignore these commands because we default to \type {every}.

{Todo: list all possible dictionaries.}

You can control the tagger by specifying what symbols and characters actually
mean, for instance:

\starttyping
\registermathfunction[𝑓]
\registermathfunction[𝑔]

% \registermathsymbol[default][en][𝐮][the vector]
% \registermathsymbol[default][en][𝐯][the vector]
% \registermathsymbol[default][en][𝖠][the matrix]

\registermathsymbol[default][en][lowercasebold]           [the vector] % [of]
\registermathsymbol[default][en][uppercasesansserifnormal][the matrix]
\stoptyping

From the language tag being used here you can deduce that this can be done per
language.

You can trace math translations with:

\starttyping
\setupnote[mathnote][location=page]
\enabletrackers[math.textblobs]
\stoptyping

which is what we used when developing these features. In a few hundred page math
book one easily gets thousands of notes.

In \type {examples-mathmeanings} you can find a lot of examples. In due time we
expect to offer more translations. The English and Swedish are for now the
benchmark. \footnote {As a proof of concept, at Bacho\TeX\ 2024, the Ukrain
translations were provided by Team Odessa, but they need some tuning.} Likely
other languages will be served by Tomáš Hala as result of courses on typesetting.
Feel free to contact all those involved in this.

\stopchapter

\startchapter[title=Tracing]

{Todo}

% \enabletrackers[structures.tags]
% \enabletrackers[structures.tags.showtree]

% \disabledirectives[structures.tags.shipout]
% \enabledirectives [structures.tags.math.standalone]
% \disabledirectives[structures.tags.math.strip]

\stopchapter

\stopdocument
