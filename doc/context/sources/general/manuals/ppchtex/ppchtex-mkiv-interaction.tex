\environment ppchtex-mkiv-style

\startcomponent ppchtex-mkiv-interaction

\chapter[txt:cooh]{Interaction}

We do support interactive texts. An interactive text is a text
that can be consulted on a computerscreen and contains many
hyperlinked textareas. This means that clicking on such an area
will result in a jump to the target area.

\startbuffer
\startchemical[axis=on,frame=on,width=5000]
  \chemical[SIX,B]
  \chemical[sub:cooh][SIX,R1,RZ1][COOH]
\stopchemical
\stopbuffer

\startfiguretext
  [left][]
  {}
  {\getbuffer}
\example[ex:interaction]
\typebuffer
\stopfiguretext

We see a new argument: the reference \type{[sub:cooh]}. This means that we
can refer from the text \goto{\chemical{COOH}}[sub:cooh] to the structure
with:

\starttyping
... text ... \goto{\chemical{COOH}}[sub:cooh] ... text ...
\stoptyping

In this definition \type{\goto} is a \CONTEXT||command.
We can also refer from the structure to a particular part of the text.

Clicking in \chemical{COOH} in the structure results in a jump to the text
that is marked with:

\starttyping
\paragraph[txt:cooh]{Substituents}

... text ... \chemical{COOH} ... text ...
\stoptyping

A combination is also possible. In that case it is necessary to mark the
reference with \type{\chemical} and to refer in the text with
\type{\gotochemical}.

\stopcomponent
