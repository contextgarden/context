\environment ppchtex-mkiv-style

\startcomponent ppchtex-mkiv-color

\chapter{Color}

In \CONTEXT\ you can colorize parts of a structure. In~\in {example}
[ex:rulecolor] and \in [ex:textcolor] we see a directive being used for the rules
but for text we just use \TEX\ commands.

\startbuffer
\startchemical[axis=on,frame=on,width=5000]
    \chemical[SIX,B,rulecolor,+R1,RZ1][red,COOH]
\stopchemical
\stopbuffer

\startfiguretext[left]{}{\getbuffer}
    \example[ex:ruledcolor]
    \typebuffer
\stopfiguretext

\startbuffer
\startchemical[axis=on,frame=on,width=5000]
    \chemical
        [SIX,B,rulecolor,+R1,color,RZ1]
        [red,{\blue COOH}]
\stopchemical
\stopbuffer

\startfiguretext[left]{}{\getbuffer}
    \example[ex:textcolor]
    \typebuffer
\stopfiguretext

\stopcomponent
