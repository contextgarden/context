\environment ppchtex-mkiv-style

\startcomponent ppchtex-mkiv-combinations

\chapter{Combinations}

Structures can be combined to more complex compounds. Moving one structure in
relation to another structure is done by \type {MOV}, \type {ROT}, \type {ADJ}
and \type {SUB}.

\placetable
{Moving and rotating.}
\starttable[ | r T | l w(2cm) | p(8cm) | ]
\HL
\VL MOV \VL Move       \VL moving a comparable structure \hbox{structure} in
                           the direction of a bond\VL\FR
\VL ADJ \VL Adjace     \VL moving another structure in the direction of the
                           $x$-- or $y$--axis, adjacent to a bond\VL\MR
\VL SUB \VL Substitute \VL moving one structure in relation to another in
                           the direction of the $x$-- or $y$--axis\VL\MR
\VL ROT \VL Rotate     \VL rotating a structure\VL\LR
\HL
\stoptable

The four keys mentioned above have different effects when they are applied to
different structures. The angle of rotation in \type {\chemical[FIVE,ROT1,B]}
differs from that in \type {\chemical [SIX,ROT1,B]}.

With the structure \type {ONE} you can use \type {MOV} but the key \type {DIR}
is also available. Both keys have the same effect but differ in spacing.
Small adjustments are possible with \type {OFF}.

\placetable
{Moving and rotating.}
\starttable[ | r T | l w(2cm) | p(8cm) | ]
\HL
\VL DIR \VL Direction \VL moving a structure in a diagonal direction      \VL \FR
\VL OFF \VL Offset    \VL moving atoms and molecules over small distances \VL \LR
\HL
\stoptable

The structure \type {CARBON} can be mirrored with \type {MIR}.

\placetable
{Mirroring.}
\starttable[ | r T | l w(2cm) | p(8cm) | ]
\HL
\VL MIR \VL Mirror     \VL mirroring a structure\VL\SR
\HL
\stoptable

We use a number to indicate the direction of a movement or the level of
rotation. These keys are closely related with the structure. Therefore they
have to be defined before bonds are drawn and texts are placed. So definition
\type {\chemical [FIVE,B,ROT1,R]} and \type {\chemical [FIVE,ROT1,B,R]} will not
have the same result. The first definition will give an undesirable result.

\startbuffer
\startchemical[frame=on,width=4000,right=3000]
  \chemical[SIX,B,MOV1,B]
\stopchemical
\stopbuffer

\startfiguretext
  [left][]
  {}
  {\getbuffer}
\example
\typebuffer
\stopfiguretext

In this example a sixring is drawn because of \type {SIX,B}. Then a movement
in the direction of bond~1 takes place and a second sixring is
drawn:~\type {B} (\type {SIX} is stil in effect).

A movement with \type {MOV} in a sixring can occur in six directions. A
movement with \type {ADJ} will take place in only four axis||directions ($x$,
$-x$, $y$, $-y$). It is a coincidence that in a sixring some of these
movements have the same effect. The example above could have been drawn with:
\type {[SIX,B,ADJ1,B]}.

Structures can be combined. It is possible for example to combine structure
\type {FIVE} with structure \type {SIX} in such a way that they have one mutual
bond. Luckily the mechanism that takes care of these kinds of combinations is
hidden for the user. In the next example you will see a sixring drawn by
\type {SIX,B}. Then a movement in the positive $x$||direction is done by
\type {ADJ1}. At last a rotated fivering is drawn: \type {FIVE,ROT3,B}.

\startbuffer
\startchemical[frame=on,width=4000,right=3000]
  \chemical[SIX,B,ADJ1,FIVE,ROT3,B]
\stopchemical
\stopbuffer

\startfiguretext
  [left][]
  {}
  {\getbuffer}
\example
\typebuffer
\stopfiguretext

\startbuffer
\startchemical[frame=on,width=5000,right=4500]
  \chemical[SIX,ROT2,B,R6,SUB1,FIVE,B,R4]
\stopchemical
\stopbuffer

\startfiguretext
  [left][]
  {}
  {\getbuffer}
\example
\typebuffer
\stopfiguretext

To go from one structure to an adjacent one is done with \type {ADJ}. Most of
the time one of these structures will have to be rotated to obtain a good
attachment. This is done by \type {ROT}. Rotations are always clockwise in
steps of 90\celsius. When a structure is attached with a bond you will have to
use \type {SUB}. Movements with \type {ADJ} and \type {SUB} take place in the
four directions of the $x$- and $y$||axis.

The next examples illustrate that the dimensions of the smaller structures
are determined by the larger structures, especially \type {SIX}. You will
notice that \type {EIGHT} has fewer possibilities than \type {SIX}.

\startbuffer
\startchemical[width=6000,left=1500,frame=on]
  \chemical[EIGHT,B,MOV1,B]
\stopchemical
\stopbuffer

\startfiguretext
  [left][]
  {}
  {\getbuffer}
\example
\typebuffer
\stopfiguretext

\startbuffer
\startchemical[width=6000,left=1500,frame=on]
  \chemical[EIGHT,B,ADJ1,SIX,B]
\stopchemical
\stopbuffer

\startfiguretext
  [left][]
  {}
  {\getbuffer}
\example
\typebuffer
\stopfiguretext

\startbuffer
\startchemical[width=6000,left=1500,frame=on]
  \chemical[EIGHT,B,ADJ1,FIVE,ROT3,B]
\stopchemical
\stopbuffer

\startfiguretext
  [left][]
  {}
  {\getbuffer}
\example
\typebuffer
\stopfiguretext

It will be clear by now that the order in which the keys are defined
makes a lot of difference. The order should be:

\starttyping
\chemical
  [structure,                              % SIX, FIVE, ...
   bonds within the structure,             % B, C, EB, ...
   bonds pointing to substituents,         % R, DR, ...
   atoms within the structure,             % Z
   subsituents attached to the structure]  % RZ, -RZ, ...
  [atoms,
   substituents]
\stoptyping

Most of the time putting structures together is done by translating and
rotating. You could automate this process. In earlier versions this was done
automatically, however this led to misinterpretations of users concerning the
positions of bonds, atoms and substituents. A structure that consists of more
than one component can best be defined per component, translations included.
Rotations should wait until the last step.

A sixring may have substituents consisting of a carbon chain. In those
situations we use \type {DIR} to build the chain.

\startbuffer
\startchemical
  [scale=small,width=6000,height=6000,frame=on]
  \chemical[SIX,SB2356,DB14,Z2346,SR36,RZ36]
    [C,N,C,C,H,H_2]
  \chemical[PB:Z1,ONE,Z0,MOV8,Z0,SB24,DB7,Z27,PE]
    [C,C,CH_3,O]
  \chemical[PB:Z5,ONE,Z0,MOV6,Z0,SB24,DB7,Z47,PE]
    [C,C,H_3C,O]
  \chemical[SR24,RZ24]
    [CH_3,H_3C]
\stopchemical
\stopbuffer

\startfiguretext
  [left][]
  {}
  {\getbuffer}
\example
\typebuffer
\stopfiguretext

Because chains have no predefined format the chains are build and positioned
as a substructure. For positioning we use the keys \type {PB} and \type {PE}.

\placetable
{Positioning.}
\starttable[ | r T | l w(4cm) | p(6cm) | ]
\HL
\VL PB:.. \VL Picture Begin \VL beginning a substructure \VL\FR
\VL PE    \VL Picture End   \VL ending a substructure    \VL\LR
\HL
\stoptable

Directly after \type {PB} you will have to define the location where the
substructure is positioned. The first following atom is centered on that
location. Always use a central atom on this location.

These keys were introduced after trying to obtain structures that are
typeset in an acceptable quality. There are different ways to define
structures. The following alternative would have resulted in:

\startbuffer
\startchemical
    [scale=small,width=6000,height=6000,frame=on]
  \chemical
    [SIX,SB2356,DB14,Z36,SR36,RZ36][N,C,H,H_2]
  \chemical
    [PB:Z1,ONE,Z0,MOV8,Z0,SB24,DB7,Z27,PE][C,C,CH_3,O]
  \chemical
    [PB:Z5,ONE,Z0,MOV6,Z0,SB24,DB7,Z47,PE][C,C,H_3C,O]
  \chemical
    [PB:Z2,ONE,Z0,MOV2,SB6,CZ0,PE][C,CH_3]
  \chemical
    [PB:Z4,ONE,Z0,MOV4,SB8,CZ0,PE][C,H_3C]
\stopchemical
\stopbuffer

\startfiguretext
  [left][]
  {}
  {\getbuffer}
\example
\typebuffer
\stopfiguretext

You may have noticed that the measurements of the structure is determined by
the substituents. The chains are not taken into account. This leads to a
consistent build-up of a structure.

The differences in outcome when using \type {SUB} in stead of \type {PB} are
very small. However compare the following formulas.

\startbuffer
\startchemical
  [width=fit,frame=on,scale=small]
  \chemical
    [SIX,ROT2,B,C,+R236,RZ23,
     SUB6,ONE,OFF1,Z0,4OFF1,SB1,Z1]
    [HO,HO,CHOH,CH_2NH_2]
\stopchemical
\stopbuffer

\startfiguretext
  [left][]
  {}
  {\getbuffer}
\example
\typebuffer
\stopfiguretext

\startbuffer
\startchemical
  [width=fit,frame=on,scale=small]
  \chemical
    [SIX,ROT2,B,C,R236,RZ23,
     PB:RZ6,ONE,Z0,3OFF1,SB1,Z1,PE]
    [HO,HO,CHOH,CH_2NH_2]
\stopchemical
\stopbuffer

\startfiguretext
  [left][]
  {}
  {\getbuffer}
\example
\typebuffer
\stopfiguretext

% The use of the key \type {PB:} might be somewhat more difficult, but the
% results are much better. In that case you should define the width yourself,
% because the substituents are not taken into account when determining the
% dimensions.

% First we will go into the key \type {OFF}. In some cases atom (\type {Z0}) in
% \type {ONE} can consist of more than one character. The reserved space for
% these characters would be insufficient and character and bond would overlap.
% When you need more space for \type {Z0} we can move bond~1, 2 and~8 by means
% of the key~\type {OFF} ('offset'). The example below will illustrate its use.

We try to predict rotations and shifts as good as possible but it might take
some trial and error, but we assume that at some poitn one has (and shares)
definitions.

\startbuffer
\startchemical
  [width=fit,frame=on]
  \chemical
    [SIX,B,C,
     ADJ1,FIVE,ROT3,SB34,+SB2,-SB5,Z345,DR35,SR4,RZ35,
     SUB4,ONE,SB258,Z0,Z28]
    [C,N,C,O,O,
     CH,COOC_2H_5,COOC_2H_5]
\stopchemical
\stopbuffer

\startfiguretext
  [left][]
  {}
  {\getbuffer}
\example
\typebuffer
\stopfiguretext

% Moving the bonds makes room for an extra character. More space was obtained
% when we would have typed \type {3OFF1}. The example looks rather complex but
% you can define it rather easy by defining its components first. Rotating
% should be done in the last stage.

% You see a new key: \type {CRZ}. This key is used to place the atom or molecule
% in one line with the bond. You could have used \type {RZ}, because you can
% influence spacing in the second argument with \type {{\,O}} in stead off
% \type {O} (spacing in mathematical mode).

We will show another example, produced in two ways. When choosing a method
you should take into account the consistency throughout your document.

\startbuffer
\startchemical
  [width=fit,height=fit,frame=on,
   scale=small]
  \chemical
    [ONE,SB15,DB7,Z057,3OFF1,MOV1,Z0,3OFF1,MOV1,
     Z017,SB1357,MOV3,Z0,MOV3,SB1357,Z013,3OFF5,
     MOV5,Z0,3OFF5,SB5,Z5]
    [C,H_2N,NH,(CH_2)_3,C,COOH,H,\SL{NH},C,COOH,H,
     (CH_2)_2,HOOC]
\stopchemical
\stopbuffer

\startfiguretext
  [left][]
  {}
  {\getbuffer}
\example
\typebuffer
\stopfiguretext

\startbuffer
\startchemical
  [width=fit,height=fit,frame=on,
   scale=small]
  \chemical [ONE,SB15,DB7,Z057,3OFF1][C,H_2N,NH]
  \chemical [MOV1,Z0,3OFF1]          [(CH_2)_3]
  \chemical [MOV1,Z017,SB1357]       [C,COOH,H]
  \chemical [MOV3,Z0]                [\SL{NH}]
  \chemical [MOV3,SB1357,Z013,3OFF5] [C,COOH,H]
  \chemical [MOV5,Z0,3OFF5,SB5,Z5]   [(CH_2)_2,HOOC]
\stopchemical
\stopbuffer

\startfiguretext
  [left][]
  {}
  {\getbuffer}
\example
\typebuffer
\stopfiguretext

\startbuffer
\startchemical
  [width=fit,height=fit,frame=on,
   scale=small]
  \chemical
    [ONE,Z0,SAVE,MOV7,SB1357,Z017,3OFF5,MOV5,Z0,
     3OFF5,MOV5,SB15,DB7,Z057,RESTORE,
     MOV3,SB1357,Z013,MOV5,3OFF5,Z0,6OFF5,SB5,Z5]
    [\SL{NH},C,COOH,H,(CH_2)_3,C,H_2N,NH,C,COOH,H,
     (CH_2)_2,HOOC]
\stopchemical
\stopbuffer

\startfiguretext
  [left][]
  {}
  {\getbuffer}
\example
\typebuffer
\stopfiguretext

\startbuffer
\startchemical
  [width=fit,height=fit,frame=on,scale=small]
  \chemical
    [ONE,Z0,MOV7,SB1357,Z017,3OFF5,MOV5,Z0,
     3OFF5,MOV5,SB15,DB7,Z057,MOV0,MOV3,SB1357,
     Z013,MOV5,3OFF5,Z0,6OFF5,SB5,Z5]
    [\SL{NH},C,COOH,H,(CH_2)_3,C,H_2H,NH,C,COOH,H,
     (CH_2)_2,HOOC]
\stopchemical
\stopbuffer

\startfiguretext
  [left][]
  {}
  {\getbuffer}
\example
\typebuffer
\stopfiguretext

Notice the use of \type {SAVE} and \type {RESTORE}. These keys enable you to save
a location in a structure and return to that location in another stage. As an
extra we will show you a combination of \type {SIX} and \type {FIVE}.

\startbuffer
\startchemical
    [width=fit,height=fit,frame=on]
  \chemical
    [SIX,DB135,SB246,Z,SR6,RZ6][C,C,N,\SR{HC},N,C,NH_2]
  \chemical
    [SIX,MOV1,DB1,SB23,SS6,Z1..3,SR3,RZ3][N,\SL{CH},N,H]
\stopchemical
\stopbuffer

\startfiguretext
  [left][]
  {}
  {\getbuffer}
\example
\typebuffer
\stopfiguretext

\stopcomponent
