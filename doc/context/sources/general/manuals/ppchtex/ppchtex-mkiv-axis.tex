\environment ppchtex-mkiv-style

\startcomponent ppchtex-mkiv-axis

\chapter{Axis}

Structures can be typeset in a frame that is divided by axis. The dimensions
of the axis and the location of the origin can be defined in the set up. The
axis and the frame can be made visible.

\startbuffer
\startchemical
  [axis=on,
   width=6000,height=4000]
\stopchemical
\stopbuffer

\startfiguretext
  [left][]{}
  {\getbuffer}
\example[ex:axis]
\typebuffer
\stopfiguretext

\startbuffer
\startchemical
  [axis=on,
   frame=on,
   width=6000,height=4000]
\stopchemical
\stopbuffer

\startfiguretext
  [left][]{}
  {\getbuffer}
\example[ex:axis]
\typebuffer
\stopfiguretext

\startbuffer
\startchemical
  [axis=on,
   frame=on,
   left=2000,right=4000]
   \chemical[SIX,B]
\stopchemical
\stopbuffer

\startfiguretext
  [left][]{}
  {\getbuffer}
\example
\typebuffer
\stopfiguretext

\startbuffer
\startchemical
  [axis=on,
   frame=on,
   width=6000,top=1000,bottom=3000]
\stopchemical
\stopbuffer

\startfiguretext
  [left][]{}
  {\getbuffer}
\example
\typebuffer
\stopfiguretext

The dimensions of the total structure determine the dimensions of the axis. When
\type {width=fit} and/or \type {height=fit} is typed the dimensions are
determined by the real dimensions so there need to be content. Your choice will
depend on how you want to place the structure in the text. As shown here, the
axis model currently uses rather large numbers. This is an inheritance of the
previous implementation and we deciced to keep it this way.

\stopcomponent
