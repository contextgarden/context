\environment ppchtex-mkiv-style

\startcomponent ppchtex-mkiv-extratext

\chapter{Extra text}

We can add text and symbols in and around structures. For example:

\startbuffer
\startchemical
  [height=4500,top=1250,width=fit,frame=on]
  \bottext
    {2,2-Dimethyl-3-ethylpentane}
  \chemical
    [ONE,Z3570,SB1357]
    [CH_3,\T{1}{H_3C},CH_3,\SR{\LT{2}{C}}]
  \chemical
    [MOV1,OFF1,Z0,SB3]
    [\T{3}{CH}]
  \chemical
    [MOV3,Z0,SB3,MOV3,Z0,MOV7,MOV7]
    [CH_2,CH_3]
  \chemical
    [OFF1,SB1,MOV1,OFF1,Z0,2OFF1,SB1,Z1]
    [\T{4}{CH_2},\T{5}{CH_3}]
\stopchemical
\stopbuffer

\startfiguretext
  [left][]
  {}
  {\getbuffer}
\example
\typebuffer
\stopfiguretext

There is a range of keys like \type{\T}. In a number of cases the arguments
are optional. Charges can be displayed in Roman by means of \type{\+} and
\type{\-} or directly by means of~\type{\1} up to~\type{\7}.

\placetable
  {Text: charges.}
  {\starttable[ | l | l | ]
  \HL
  \VL \type{\+{number}} \VL positive charge in Roman   \VL\FR
  \VL \type{\-{number}} \VL negative charge in Roman   \VL\MR
  \VL \type{\1}         \VL I (without sign)           \VL\MR
  \VL \type{\7}         \VL II, III, IV, V, VI and VII \VL\LR
  \HL
  \stoptable}

A charge is centered above the atom. For example:

\startbuffer
\placeformula
  \startformula
      \chemical{S}
      \chemical{+}
      \chemical{O_2}
      \chemical{GIVES}{violent}
      \chemical{\+{4}{S}\-{2}{O_2}}
  \stopformula
\stopbuffer

\typebuffer

will result in:

\getbuffer

Compare this with:

\startbuffer
\placeformula
  \startformula
    \startchemicalformula
      \chemical{S}
      \chemical{+}
      \chemical{O_2}
      \chemical{GIVES}{violent}
      \chemical{\+{4}{S}\-{2}{O_2}}
    \stopchemicalformula
  \stopformula
\stopbuffer

\typebuffer

which gives:

\getbuffer


\endinput

If we want to repeat a number of atoms or molecules we can define an
(endless) range with \type{\[} and \type{\]}. Both arguments are optional as
is shown in the example of \kap{PTFE} of Polytetrafluorethane, better knowns
as Teflon.

\startbuffer
\startchemical[width=fit,frame=on]
  \chemical
    [ONE,ZT5,SB5,OFF1,Z0,OFF1,SB1,MOV1,SB5,OFF1,Z0,OFF1,SB1,ZT1]
    [\[,CF_2,CF_2,\]{\sl n}]
\stopchemical
\stopbuffer

\startfiguretext
  [left][]
  {}
  {\getbuffer}
\example
\typebuffer
\stopfiguretext

\placetable
  {Text: repeating.}
  {\starttable[ | l | l | l | ]
   \HL
   \VL \type{\[{bottom}} \VL \type{\[{top}{bottom}} \VL right repeating sign \VL\FR
   \VL \type{\]{bottom}} \VL \type{\]{top}{bottom}} \VL left repeating sign  \VL\LR
   \HL
   \stoptable}

There is no problem of placing texts
on the left, right, top or bottom of the atoms or molecules.
If we preceed the keys \type{\L}, \type{\R}, \type{\T} and \type{\B}
by \type{\X} the distance from text to atoms is somewhat smaller.

\placetable
  {Text: around an atom.}
  {\starttable[ | l | l | ]
   \HL
   \VL \type{\L{text}} \VL text left   \VL\FR
   \VL \type{\R{text}} \VL text right  \VL\MR
   \VL \type{\T{text}} \VL text top    \VL\MR
   \VL \type{\B{text}} \VL text bottom \VL\LR
   \HL
   \stoptable}

Logical combinations of these keys are also possible. A key to centre text is
also available.

\dontleavehmode\hbox to \hsize
  {\def\sample#1{\chemical{#1{\oplus}{\ruledhbox{\ttx\string#1}}}}%
   \sample\TL\hss\sample\L \hss\sample\LC\hss\sample\BL\hss
   \sample\TR\hss\sample\R \hss\sample\RC\hss\sample\BR\hss
   \sample\LT\hss\sample\T \hss\sample\RT\hss\sample\LB\hss
   \sample\B \hss\sample\RB}

\dontleavehmode\hbox to \hsize
  {\def\sample#1{\chemical{\X#1{\oplus}{\ruledhbox{\ttx\string\X\string#1}}}}%
   \sample\TL\hss\sample\L \hss\sample\LC\hss\sample\BL\hss
   \sample\TR\hss\sample\R \hss\sample\RC\hss\sample\BR\hss
   \sample\LT\hss\sample\T \hss\sample\RT\hss\sample\LB\hss
   \sample\B \hss\sample\RB}

In some cases you will need what we may call {\em smashed} text.

\placetable
  {Text: smashed text.}
  {\starttable[ | l | l | ]
   \HL
   \VL \type{\SL{text}} \VL left align  \VL\FR
   \VL \type{\SM{text}} \VL centre      \VL\MR
   \VL \type{\SR{text}} \VL right align \VL\LR
   \HL
   \stoptable}

An example is given below. The text is centred around the first character.

\startbuffer
\startchemical[frame=on]
  \chemical[SIX,B,R36,RZ36][\SL{COOH},\SL{COOH}]
\stopchemical
\stopbuffer

\startfiguretext
  [left][]
  {}
  {\getbuffer}
\example
\typebuffer
\stopfiguretext

We can place text above, under or in the middle of
structures with the keys \type{\toptext}, \type{\midtext}
and \type{\bottext}. The exact position is determined by the
height and depth of the structure.

\startbuffer
\placeformula
  \startformula
    \startchemical[width=fit]
      \chemical[SIX,B,Z1,MOV1,B][\hbox{$\bullet$}]
      \toptext{{\sl trans}-Decalin}
    \stopchemical
    \hskip 24pt
    \startchemical[width=fit]
      \chemical[SIX,B,Z12,MOV1,B][\hbox{$\bullet$},\hbox{$\bullet$}]
      \bottext{{\sl cis}-Decalin}
    \stopchemical
  \stopformula
\stopbuffer

\typebuffer

Both {\em Decalin} formulas look like this:

\getbuffer

\stopcomponent
