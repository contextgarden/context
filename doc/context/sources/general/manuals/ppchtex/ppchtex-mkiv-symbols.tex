\environment ppchtex-mkiv-style

\startcomponent ppchtex-mkiv-symbols

\chapter{Symbols}

There are some symbols that can be used to display reactions. The reaction
below is typed by:

\startbuffer
\setupchemical
  [size=small,
   scale=small,
   width=fit,
   height=5500,
   bottom=1500]

\hbox
  {\startchemical
     \chemical[SIX,B,ER6,RZ6][O]
   \stopchemical
   \startchemical
     \chemical[SPACE,PLUS,SPACE]
   \stopchemical
   \startchemical
     \chemical[FIVE,ROT4,B125,+SB3,-SB4,Z4,SR4,RZ4][N,H]
   \stopchemical
   \startchemical
     \chemical[SPACE,GIVES,SPACE][?]
   \stopchemical
   \startchemical
     \chemical[SIX,B,EB6,R6,SUB4,FIVE,ROT4,B125,+SB3,-SB4,Z4][N]
   \stopchemical
   \startchemical
     \chemical[SPACE,PLUS,SPACE,CHEM][H_2O]
   \stopchemical}
\stopbuffer

\typebuffer

The \type{\hbox} is necessary to align the structures. The symbols
\type{GIVES} and \type{PLUS} need no further explanation. With \type{SPACE}
more room can be created between the structures and symbols.

\placefigure
  [hier]
  [fig:reactie]
  {}
  {\getbuffer}

An equilibrium can be displayed with \type{EQUILIBRIUM}. Over \type{GIVES}
and \type{EQUILIBRIUM} you can place text. In the example the text is just a
'?'. In addition \type{MESOMERIC} is also available. Braces used for
displaying complexes can be created with \type{OPENCOMPLEX} and
\type{CLOSECOMPLEX}.

\stopcomponent
