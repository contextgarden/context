% language=us runpath=texruns:manuals/musings

\startcomponent musings-sideeffects

\environment musings-style

\startchapter[title={Side effects}]

In the process of extending \TEX\ into \LUAMETATEX\ I occasionally run into what
I consider side effects. Some might argue that we're seeing a bug but personally
I never consider what I observe as \quotation {interestingly weird} to be a bug.
After all, if it really was an issue Don Knuth would already have run into it and
dealt with it accordingly.

Mid|-|2023 Mikael Sundqvist and I were reviewing the upgraded multi|-|line display
math mechanism and in particular we were looking at how it behaves with for
instance side floats. At some point we wondered if such a formula should follow
the shape. This is not that trivial because where regular text is in most cases
sort of grid|-|based and therefore has well|-|defined line heights, display math is
way less predictable. This means that setting up hanging indentation based on
lines will often lead to bad results. I did actually check if the par builder
could be adapted to work with a height specification but this is not trivial, if
only because one then needs to keep track of (intermediate) line heights while at
the same time one doesn't know if the interline spacing changes, which actually
is something a user can set up in \CONTEXT\ display math. Setting the hang or
shape comes before the content so we also have an out|-|of|-|order situation to
deal with. Of course placing a formula number in the margin or below a formula
also interferes.

Because formulas following a shape looked weird anyway (think for a moment what
mid|-|aligned display math should look like) in the end we decided that a display
formula in a side float situation is just an unbroken vertical box that if needed
will extend below the allotted hang space. This is not that different from what
\TEX\ does when we would use display math (which we don't in \CONTEXT, where we
follow a different approach) because \TEX\ just assumes that a display formula
counts as three lines progression in a hanging (or par shape) continuation.

\startbuffer
\dontleavehmode
\hbox{\strut one}
\break
\hbox{\strut two}
\par
\the\prevgraf
\stopbuffer

\typebuffer

This example shows two boxes making a single paragraph and the number of lines
reported is two:

\dbox{\forgetall \showmakeup[line] \getbuffer}

That number is not always reliable as we can see from:

\startbuffer
\dontleavehmode
\hbox{\strut one}
\break
\vbox{\strut two\par \strut three}
\par
\the\prevgraf
\stopbuffer

\typebuffer

Again we get two, which indicates that a line is a box and not a text line which
is why for standard \TEX\ display math a value of three is added, assuming simple
single line formulas of not that much height and with spacing around it that make
it somewhat similar to three lines.

\dbox{\forgetall \showmakeup[line] \getbuffer}

The way these three lines are communicated is via \type {\prevgraf} which is a
property of the currently worked on list. When the par builder kicks in, its
current value is taking as starting point for the first line number and line
numbers are used to pick up the right indentation (from hang or shape). This
means that when the hang or shape parameters are not reset, as in the case of a
mid paragraph display formula in \TEX, we pick up where we left. So, that made me
wonder what happens if we set the \type {\prevgraf} ourselves? As with \type
{\prevdepth}, which has its own interesting side effects, we can get some
surprising outcomes. For showing this we use the following macro:

\def\Test#1#2%
  {\begingroup
   \darkblue
   \blank
   \hsize      9cm
   \hangafter -3
   \hangindent 1cm
   \begstrut
   \ifcase#1\relax
     \samplefile{knuthmath}
   \or
     \splitsamplefile{knuthmath}{4}{3}{...}
   \or
     \splitsamplefile{knuthmath}{10}{5}{...}
   \or
     \splitsamplefile{knuthmath}{20}{5}{...}
   \fi
   \endstrut
   \prevgraf   #2
   [\number\prevgraf]
   \blank
   \endgroup}

\starttyping
\def\Test#1#2%
  {\begingroup
   \hsize      9cm
   \hangafter -3
   \hangindent 1cm
   sample data % from a sample file
   \prevgraf   #2
   [\number\prevgraf]
   \par % not needed in ctx
   \endgroup}
\stoptyping

\start \bitwiseflip \normalizeparmode -\limitprevgrafnormalizeparcode % begin of flip

So let's start with the zero case. We use a quote from Mathematical Writing, a
course by Don Knuth that Mikael and I agree we would have failed to pass:

\Test{0}{0}

The output is as we expect: we get three indented lines. So what happens when we
set the \type {\prevgraf} to one?

\Test{0}{1}

We're good, so let's try two:

\Test{0}{2}

Going up from there we see that values larger than three seemingly get ignored,
so a safe thousand produces:

\Test{0}{1000}

I then became curious what happens with effectively zero values but that is hard
because when you set \type {\prevgraf} negative assignments give an error. That
made me wonder what would happen with the maximum positive value, given that I'd
seen the code involved doing additions:

\Test{1}{2147483647} % {\maxcount}

Okay, that looks interesting! So what if we go a bit below that?

\Test{3}{2147483645}

\stop % end of flip

We now come to the question: \quotation {Is this really what Don Knuth had in
mind?} For sure the answer is \quotation {yes} if we can come up with a use
case. However, because once a user knows that this feature is there, and because
we cannot be sure (without examining the code) if a wrap around can make \TEX\
fail, it is best to just catch this case. In order to remain compatible we do this
via an option: we set the third bit in \type {\normalizeparmode}. In \CONTEXT\
speak:

\startbuffer[flip]
\bitwiseflip \normalizeparmode \limitprevgrafnormalizeparcode
\stopbuffer

\typebuffer[flip]

\start % begin of flip

\getbuffer[flip]

As you can see below, we're fine now.

\Test{0}{2147483647} % {\maxcount}

\stop % end of flip

Can we come up with a use case? Sure. In plain \TEX\ there is a macro that
will show the word snippets on the console:

% \hangafter 1 \samplefile{knuthmath} \prevgraf 2147483647

\startbuffer
\def\showhyphens#1%
  {\setbox0\vbox
     {\parfillskip  0pt
      \hsize        \maxdimen
    % \tenrm
      \pretolerance-1
      \tolerance   -1
      \hbadness     0
      \showboxdepth 0
    % \
      #1}}
\stopbuffer

\typebuffer

\pushoverloadmode \getbuffer \popoverloadmode

Now watch the extreme \type {\hsize} here. We clip the one line result to the
text width. One would expect a reasonable sample text to fit on that line and
indeed it does but the tolerance values combined with the lack of a final filler
will lead to underfull box messages.

\startlinecorrection
\showhyphens{\samplefile{knuthmath}}
\ruledhbox{\clip[width=\textwidth,height=\lineheight]{\strut\box0}}
\stoplinecorrection

If we use \type {.2\maxdimen} instead we see this:

\startlinecorrection
\pushoverloadmode \divide \maxdimen by 5 \popoverloadmode
\showhyphens{\samplefile{knuthmath}}
\ruledhbox{\clip[width=\textwidth,height=\lineheight]{\strut\box0}}
\stoplinecorrection

Here comes our use case. We limit the text because here we get multiple lines.
Interestingly, we don't even need to set \type {\hangindent} to some value.

\startbuffer
\def\showhyphens#1%
  {\setbox0\vbox
     {#1
      \prevgraf 2147483647
     %\showboxdepth 0
      }}
\stopbuffer

\typebuffer

or in \CONTEXT\ speak:

\starttyping
\def\showhyphens#1{\setbox\scratchbox\vbox{#1\prevgraf\maxcount}}
\stoptyping

\pushoverloadmode \getbuffer \popoverloadmode

\showhyphens{\splitsamplefile{knuthmath}{4}{3}{...}}

So one can argue that letting \type {\prevgraf} wrap around is a feature
that permits efficient trickery.

\startlinecorrection
\dontleavehmode \box0 \vskip-\lineheight
\stoplinecorrection

It is this particular neat use case that turn this \quote {wrap around} side
effect into a proper feature, although in \CONTEXT\ \LMTX\ we will likely enable
the catch. But the fact that on the one hand setting \type {\prevgraf} has a
check for a negative value, while on the other hand the par builder permits
\quote {wrapping around} might be an argument for some \quotation {grunt­ing
noise}.

\subsubject{post scriptum}

In \TEX\ there is a difference between these two cases:

\starttyping
some text     \par
$$ formula $$ \par
some text     \par
\stoptyping

and

\starttyping
some text     % stay in par
$$ formula $$ % stay in par
some text     \par
\stoptyping

In the first case we have three paragraphs and if we use some hanging indentation
or par shape it only covers the current paragraph. In the second case we stay
in the paragraph and therefore the hang or shape applies to all three. Now here comes
the interesting bit.

\def\Test#1#2%
  {\begingroup
   \showmakeup[line]
   \everypar           {}
   \mathdisplaymode    0
   \mathdisplayskipmode0
   \abovedisplayskip   #1\lineheight
   \belowdisplayskip   #1\lineheight
   \hangindent         1cm
   \hangafter          -7
   Many readers will skim over formulas on their first reading of your exposition.
Therefore, your sentences should flow smoothly when all but the simplest formulas
are replaced by \quotation {blah} or some other grunting noise.
 % 3 lines
   $$ x = 1 $$        % 3 lines
   \prevgraf\numexpr\prevgraf+#2\relax
   Many readers will skim over formulas on their first reading of your exposition.
Therefore, your sentences should flow smoothly when all but the simplest formulas
are replaced by \quotation {blah} or some other grunting noise.
 % 3 lines
   \par
   \endgroup}

Compare

\Test{0}{0}

With

\Test{1}{0}

and

\Test{2}{0}

As I mentioned before, a display formula counts as three lines and that is
communicated to the par builder by adapting \type {\prevgraf}. This is actually
independent of the amount of spacing we add so you can take it into account when
setting up such a hang. It is one of these cases where more clever automatism
makes little sense.

Because spacing around formulas can be tricky in \CONTEXT\ we always managed it
differently. For that reason in \LUATEX\ we added some more engine control over
the scenarios, for instance in relation to \type {\baselineskip} related catches.
We just default to no spacing at all. In \LMTX\ we emulate display mode even
more different and also use the \LUAMETATEX\ option to make the dollar driven
display mode equal to inline, in order to force users to use the proper
mechanisms. In \CONTEXT\ we also have some plugins into \type {\everypar} that
(re)set the shapes so in order to produce the above I had to bypass that. I can't
speak for other macro packages but in plain a test looks like this:

\starttyping
\abovedisplayskip 0pt
\belowdisplayskip 0pt
\hangindent       1cm
\hangafter        -7
Many readers will skim over formulas on their first reading of your exposition.
Therefore, your sentences should flow smoothly when all but the simplest formulas
are replaced by \quotation {blah} or some other grunting noise.
 % 3 lines (@12pt)
$$ x = 1 $$        % 3 lines
Many readers will skim over formulas on their first reading of your exposition.
Therefore, your sentences should flow smoothly when all but the simplest formulas
are replaced by \quotation {blah} or some other grunting noise.
 % 3 lines
\stoptyping

In this document I cheated by adding:

\starttyping
\everypar            {}
\mathdisplaymode     0
\mathdisplayskipmode 0
\stoptyping

So now we know one reason why \type {\prevgraf} is really there and why wraparound
is not important. Of course one can wonder why that (current list) property
can be set at all by a user. Here's one:

\Test{0}{-2}

How did we get this? The par builder kicks in when we handle display math but
it retains the shape. That means that after the formula is done we have a
positive value of \type {\prevgraf} and that means we can subtract from it
without going negative (and triggering an error):

\starttyping
Many readers will skim over formulas on their first reading of your exposition.
Therefore, your sentences should flow smoothly when all but the simplest formulas
are replaced by \quotation {blah} or some other grunting noise.

$$ x = 1 $$
\prevgraf\numexpr\prevgraf-2\relax % etex
Many readers will skim over formulas on their first reading of your exposition.
Therefore, your sentences should flow smoothly when all but the simplest formulas
are replaced by \quotation {blah} or some other grunting noise.

\stoptyping

As it is not a (engine or user) register we can't advance and because it operates
on the current list grouping is irrelevant, so for pure \TEX\ one has to do
something:

\starttyping
{\count2=\prevgraf \advance\count2 by -2 \prevgraf=\count2} % tex
\stoptyping

So, if really needed one can typeset the formula and from its dimensions and
spacing above and below derive the correction. Here we just subtract two. There
is no way that Mikael can convince me to add a fragile feature like this to
\CONTEXT, and I know he won't. We anyway have different nuts (or nut math) to
crack. Also, we promote a more spacy source setup so users will definitely add
empty lines (\type {\par} equivalents) before and after formulas, which makes all
the above fail. But out of curiosity, maybe we should get the latest TAOCP
collection and check where display math is used in a hang or shape context.

\stopchapter

\stopcomponent
