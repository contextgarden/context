% language=us runpath=texruns:manuals/musings

% The basic format uses some funny symbols so we need to load:

\usemodule[math-verbatim]

% This is a stand-alone chapter of the musing series.

\startcomponent musings-basic

\environment musings-style

\hyphenation{typo-graph-ers}
\hyphenation{regis-ters}

\startdocument

% So we have a dedicated title page. This bit comes straight from the
% \LUAMETAFUN\ manual, the chapter about Perlin noise (Keith McKay,
% Mikael Sundqvist, Hans Hagen).

\protected\frozen\instance\def\MyTeX
  {\dontleavehmode
   \begingroup
   \scratchdimen\scaledfontcharwd\font`M%
   T%
   \kern-.1667\scratchdimen
   \lower.415\exheight\hbox{E}% but on the average this looks better
   \kern-.125\scratchdimen
   X%
   \endgroup}

\startluacode
    function MP.MyFunction1(v) return v/2, v, v/4 end
    function MP.MyFunction2(v) return v, v/2, v/4 end
    function MP.MyFunction3(v) return v/2, v/4, v end
\stopluacode

\startMPdefinitions
vardef OneTeX (expr bmp, clrfnc, n, what) =
    save p ; picture p, q ;
    q := image ( draw
        (
            lmt_outline [
                kind = "outline",
                text = "\strut \bf \MyTeX",
            ]
        ) xsized .9PaperWidth
        withpattern image (
            draw lmt_noise [
                bytemap       = bmp,
                nx            = 500,
                ny            = 500,
                nz            = 3,
                iterations    = 1,
                frequency     = 0.01,
                amplitude     = 1,
                persistence   = 0.5,
                lacunarity    = 2.0,
                minimum       = 0,
                maximum       = 255,
                colorfunction = clrfnc,
                method        = "angle",
            ] xsized PaperWidth
            ;
        ) ;
    ) ;
    p := image (
        draw
            textext(what)
            ysized .05PaperHeight
            shifted lrcorner q
            shifted (-0.075PaperWidth,0.02PaperHeight)
            withcolor white ;
        draw q ;
    ) ;
    setbounds p to boundingbox q ;
    p
        shifted - llcorner p
        shifted (.05PaperWidth,0)
        shifted (0,n*PaperHeight/3+.05PaperWidth)
enddef ;
\stopMPdefinitions

% The main text is in Pagella about we do the title page in good old Latin
% Modern as derived from Computer Modern.

\start \setupbodyfont[modern,12pt]
\startMPpage
    StartPage ;
        fill Page withcolor .5 ;
        draw image (
            draw OneTeX(1,"MyFunction1",0,"\tttf 89") ;
            draw OneTeX(2,"MyFunction2",1,"\tttf 82") ;
            draw OneTeX(3,"MyFunction3",2,"\tttf 78") ;
        ) ;
        setbounds currentpicture to Page ;
    StopPage ;
\stopMPpage
\stop

% A few acronyms:

\logo[SAIL]{SAIL}
\logo[MESA]{MESA}

% And off we go!

\starttitle[title={Back to the basics}] % Back-o-\TeX

\startluacode
-- primitive("lineskip",assign_glue,glue_base+line_skip_code);

local gsub, gmatch = string.gsub, string.gmatch

local version = 2

local primitives = table.load("musings-basic-tmp.lua")

if primitives and primitives.version == version then

    context.writestatus("document","")
    context.writestatus("document","using saved data file 'musings-basic-tmp.lua'")
    context.writestatus("document","")

else

    function document.filterprimitives(filename)
        local data = io.loaddata(resolvers.findfile(filename))
        local list = { }
        for primitive, cmd, chr in gmatch(data,"primitive%(\"([^\"]-)\",([^,]+),([^%)]+)") do
            list[#list+1] = {
                (gsub(primitive,"%s","")),
                (gsub(cmd,      "%s","")),
                (gsub(chr,      "%s","")),
            }
        end
        return list
    end

    primitives = {
        old     = document.filterprimitives("musings-basic-old.web"), -- version   0.25
        new     = document.filterprimitives("musings-basic-new.web"),
        version = version,
    }

end

function document.showprimitives(what)
    local list = primitives[what]
    if list then
        context.starttabulate { "|T|T|T|" }
        local oldcmd = false
        for i=1,#list do
            local l = list[i]
            local prm = l[1]
            local cmd = l[2]
            local chr = l[3]
            if oldcmd and oldcmd ~= cmd then
--                context.TB()
            end
            oldcmd = cmd
            context.NC() context.tex(prm)
            context.NC() context(cmd)
            context.NC() context(chr)
            context.NC() context.NR()
        end
        context.stoptabulate()
    end
end

function document.showcommands(what)
    local list = primitives[what]
    if list then
        local result = table.setmetatableindex("number")
        for i=1,#list do
            local l = list[i]
            local cmd = l[2]
            result[cmd] = result[cmd] + 1
        end
        context.starttabulate { "|T|r|" }
        for cmd, n in table.sortedhash(result) do
            context.NC() context(cmd)
            context.NC() context(tostring(n))
            context.NC() context.NR()
        end
        context.stoptabulate()
    end
end

function document.showcombined()
    local both = table.setmetatableindex("table")
    for k, v in next, primitives.old do
        local b = both[v[2]]
        b.old = (b and b.old or 0) + 1
    end
    for k, v in next, primitives.new do
        local b = both[v[2]]
        b.new = (b and b.new or 0) + 1
    end
    context.starttabulate { "|T|c|c|" }
    for cmd, n in table.sortedhash(both) do
        context.NC() if n.old and n.new then context.bold(cmd) else context(cmd) end
        context.NC() if n.old then context(n.old) end
        context.NC() if n.new then context(n.new) end
        context.NC() context.NR()
    end
    context.stoptabulate()
end

table.save("musings-basic-tmp.lua",primitives)

local commandnames = tokens and tokens.commands
local primitives   = token.getprimitives()

-- inspect(commandcount)
-- inspect(commandhash)

function document.showluametatex()
    local commandhash  = { }
    local commandcount = { }
    for k, v in next, token.getcommandvalues() do
        commandcount[v] = 0
    end
    for i=1,#primitives do
        local p = primitives[i]
        local cmd = commandnames[p[1]]
        commandcount[cmd] = commandcount[cmd] + 1
     -- commandhash[p[3]] = p
    end
    context.starttabulate { "|T|r|" }
    for cmd, n in table.sortedhash(commandcount) do
        context.NC() context(cmd)
        context.NC() if n ~= 0 then context(tostring(n)) end
        context.NC() context.NR()
    end
    context.stoptabulate()
end

\stopluacode

\startsection[title=Introduction]

When Mikael told me that there were enhanced lectures by Don Knuth, pushed
on-line at Stanford, I decided to have a look at it. \footnote {He got the link
to the upgraded videos from Barbara Beeton who was present at one or more of
these lectures.} We're talking of two series: \quotation {Advanced \TEX arcana}
(1981, so between the \SAIL \footnote {\SAIL\ refers to the \quotation {Stanford
Artificial Intelligende Laboratory}, and references to this acronym refer to the
mainframe at that place and its operating system variant. On Wikipedia search for
\type {WAITS} as it also refers to early days media usage.} and \TEX82 version)
and \quotation {The Internal Details of \TEX82} (during early \TEX82\
development). After all, it's a nice distraction when one is supposed to be in
documentation mode. It is also a good retrospective on the period around the 80's
when \TEX\ was actively developed, tested on real documents, and when more people
became interested in using it, evolved. So I sat down (or walked around),
listened, and took some notes. \footnote {This wrapup is part of the musing
series but rendered independently because of its length.}

When watching these videos one has to keep in mind that the version of \TEX\ and
its default format plain \TEX\ were not yet finalized, even if there's mentioning
of the books being released any time (next year). Rather soon one starts to
notice that there are references to primitives and macros that predate the
standard \TEX\ engine and plain format. For instance instead of \quote {plain}
there is \quote {basic}. Some examples and concepts assume a different \TEX\ than
one knows today and it's interesting to where all we know and use now came from.
These lectures are part of the development cycle of many years and also a clear
demonstration of how software development took place: in an academic setting
without commercial pressure. With user feedback, discussions, and no problems
going back to the drawing board. It all resulted in \TEX\ as we know it today,
and although there have been extensions (enhancements) the principles remain the
same. \footnote {The \LUA\ project has a similar vibe which is among the reasons
that we've chosen it as the extension language for \LUATEX.}

Before we continue, let's guess about the timeline here; please correct me if I'm
wrong.

\startitemize[packed,loose]
\startitem
    In the late 70's Don Knuth started working on what became \TEX. The reasons
    for this (quality) are explained in various places but of course making sure
    that his books look great was the main objective.
\stopitem
\startitem
    The program was written in \SAIL, running on a Digital Equipment
    infrastructure, but later \PASCAL\ was used. There are references to other
    hardware setups (at other universities) and in one talk SUN workstations are
    mentioned which illustrates the move from terminals to more integrated high
    resolution setups.
\stopitem
\startitem
    The program was of course initially meant for producing books, and with that
    in mind it started out as a production tool. Much was hard-coded but some was
    implemented in a macro package called \quote {basic}. In these years we see
    the language evolve to be more flexible: there came additional programming
    related features and hard coded assumptions became configurable.
\stopitem
\startitem
    At some point a reimplementation was started, this time in \PASCAL, generated
    from \WEB\ files. A more detailed documentation of how things work
    conceptually are part of the process and stepwise building up the program
    paradigm. The mature \WEB\ suite is part of the outcomes of the \TEX\ and
    \METAFONT\ project. \footnote {Because \PASCAL\ became less popular and was
    not available on all platforms, at least not in compatible ways, Knuth later
    switched to \CCODE\ and therefore \CWEB. The \WEBC\ conversion introduced later
    converts \PASCAL\ code into \CCODE\ which then compiles into the programs
    distributed in for instance \TEXLIVE.}
\stopitem
\startitem
    The arcana presentations took place around that time, so they cover the
    prototype(s) as well as the early \TEX82\ versions. Between 1983 and 1989
    \TEX\ became more and more stable and eventually it was declared finished and
    frozen, except for bugs periodically being fixed.
\stopitem
\startitem
    Then \TEX82\ started being used and a period of development, documentation and
    experimenting started. Various students were involved in subsystems.
\stopitem
\startitem
    The presentations about the internal details happen in the middle of this
    evolution, or maybe a better term is transition. There is still talk of a
    basic format, some primitives will later change, the book we mention here
    dates from that time, the books mentioned being written are the multi-column
    set about \TEX, \METAFONT\ and the fonts. \footnote {This first book,
    discussing digital typography, basic \TEX\ and infant \METAFONT, has header
    lines wrapped in a frame. In those days it was actually quite common (in
    educational documents) to put rules around things, maybe just because it
    could be done. It might have been a left-over from the typewriter days when
    little was possible and with upcoming systems suddenly one could draw rules.
    The later \TEX book series looks way better! But still, quite often
    extensively framed tables show up in documents produced by \TEX. Maybe
    because the core \TEX\ engine is pretty much limited to glyphs and rules out
    of the box.}
\stopitem
\startitem
    From 1982 to 1989 the versions evolve and in the tape dumps of the
    by then retired \SAIL\ machines we see less and less changes. We also
    see macro packages pop up and evolve. The fonts mature as well.
\stopitem
\stopitemize

The above is more or less noticeable in the presentations. For instance, the many
references to \quotation {loading basic} made me curious to what this prehistoric
\TEX\ format actually looked like, if only to figure out what references to \type
{\jpar} actually mean (given the context one can guess here). The macros,
actually a relative small set compared to plain, and tiny compared to for
instance \CONTEXT\ \MKXL, can be found in \quotation {\TEX\ and \METAFONT, New
Directions in Typesetting}, published by the now defunct Digital Equipment, the
company that made the main frame and mini computers that \TEX\ was developed on
and that also are mentioned (and used) in the presentations.

Watching all that comes with some nostalgic feelings, because I grew up with
those machines too (DEC 10 and 20, as well as VAX), did quite a bit of \PASCAL\
and \MODULA2 programming, used the line printers, slow tube terminals, went from
300 to 1200 baud modems, and even played with these GIGI terminals. Being totally
unaware of something \TEX\ running on these machines I even wrote some programs
that made pages from \ASCII\ input, including generating tables of contents,
doing some itemization, move around some space for glued in images, and page
numbering.

So, a lot of bells ring when seeing the black and white videos, filmed just
before color video entered academia, eventually to be replaced by the Internet.
And yes, the first personal computers also showed up, but again, no \TEX.
Anyways, looking back it is amazing that something \TEX\ took off so well,
especially given what users nowadays take for granted with respect to
performance, editing and previewing, and are sometimes willing to complain about.

But, let's move beyond the sentimental reflections and have a look at this basic
format because that kind of pictures the landscape. This is not a tutorial
so no details will be explained but I'll revive some of what can be read in this
first \TEX\ manual because it's not something you can pick up in a bookshop.
The book actually is a nice read, with plenty of humor sprinkled in.

\stopsection

\startsection[title=Some observations]

We will use the basic format to explain a bit what this first version was about
but before we come to that it might be fun to filter out some distinctive
differences with today's \TEX. Let's start with a quote: \footnote {There is a
list of named characters in the manual but they are not defined in the basic
format, so I guess that there is an extra file that defines them.}

\startquotation
    Those of you who wish to define control sequences should know that \TEX\ has
    further rules about them, namely that many different spellings of the same
    control sequence may be possible. This fact allows \TEX\ to handle control
    sequences quite efficiently; and \TEX's usefulness is not seriously affected,
    because new control sequences aren't needed very often. A control sequence of
    the first kind (i.e., one consisting of letters only) may involve both upper
    case and lower case letters, but the distinction between cases is ignored
    after the first letter. Thus \type {\TEX} could also be typed \quotation
    {\type {\TEx}} or \quotation {\type {\TeX}} or \quotation {\type {\Tex}}
    \emdash\ these four have the same meaning and the same effect. But
    \quotation {\type {\tex}} would not be the same, because there is a case
    distinction on the first letter. (Typing \quotation {\type {\gamma}} results
    in \gamma, but \quotation {\type {\Gamma}} or \quotation {\type {\GAMMA}}
    results in \Gamma.)
\stopquotation

In one presentation Knuth explains that \PASCAL\ has some limitations on the
length of an identifier: eight, and therefore he played safe by limiting them to
seven unique initial characters in his code. Maybe that inspired him to come up
with the feature mentioned here. The question is why it was done this way. I
assume that efficiency here refers to hashing and resolving control sequences
because normalizing takes runtime too. Once stored in the format (or memory) the
control sequence is a token and therefore just a number, so any length or case
property is gone. It is one of the reasons why \TEX\ is fast and efficient!
However, keep in mind that tokens as concept were not in the first
implementation, there it was about sequences of characters (which then is
slower).

But more interesting is the assumption that not that many new control sequences
are needed. Tell that to a 2025 \TEX\ user or macro package writer: it just
wouldn't work out today. Of course this became clear pretty soon so this feature
was dropped in the follow up. It was mentioned in a dangerous bend section that
when a reader dared reading that, lead to a next bend:

\startquotation
    Another rule takes over when there are seven or more letters after the
    escape: all letters after the seventh are replaced by \quotation {x}, and
    then groups of eight letters are removed if necessary until at most 14
    letters are left. Thus \type {\underline} is the same as \type {\underlixx};
    and it is also the same as \type {\underlinedsymbols} or any other control
    sequence that starts with \type {\u} followed by n or N, then d or D, then e
    or E, then r or R, then l or L, then i or I, then 2 or 10 or 18 or 26 or
    \cdots\ letters. But \type {\underline} is not the same as \type
    {\underlines}, because these two control sequences don't have the same length
    modulo 8.
\stopquotation

I'm sure you \quotation {get this} in one read, but try to explain that to a
confused user who defined a ton of macros after having managed to bump \TEX's
memory to the extremes of those days. \footnote {Just imagine explaining \type
{\big}, \type {\Big}, \type {\bigg} and \type {\Bigg} versus the impossible \type
{\Big}, \type {\BIg} and \type {\BIG} when only the first capital is
distinctive.} But, as usual, Don's wit kicks in when he makes us go into the next
dangerous bend:

\startquotation
    \unknown\ Thus the total number of distinct control sequences available is exactly

    \startformula
        128 + 52 • 26 + 52 • 26^2 + 52 • 26^3 + 52 • 26^4 + 52 • 26^5 + 8 •
        52 • 26^6 = 129151507704
    \stopformula

    that should be enough \unknown
\stopquotation

In the \WEB\ file we can read this (and best keep this in mind when we progress
in this wrap-up):

\startquotation
    The present implementation has a long ancestry, beginning in the summer
    of~1977, when Michael~F. Plass and Frank~M. Liang designed and coded a
    prototype based on some specifications that the author had made in May of
    that year. This original proto\TeX\ included macro definitions and elementary
    manipulations on boxes and glue, but it did not have line-breaking,
    page-breaking, mathematical formulas, alignment routines, error recovery, or
    the present semantic nest; furthermore, it used character lists instead of
    token lists, so that a control sequence like \type {\halign} was represented
    by a list of seven characters.
\stopquotation

Watch the fact that there was no token list (as we know it now) yet but a
sequence of characters. So storing a macro name was costly. Of course a list of
records with two numbers (the token and the pointer to the next token) also takes
space. This might be the reason why we have the collapsing of names mentioned
above: it had to fit into memory (and format files, which are kind of memory
dumps) efficient.

\startquotation
    A complete version of \TeX\ was designed and coded by the author in late 1977
    and early 1978; that program, like its prototype, was written in the \SAIL\
    language, for which an excellent debugging system was available. Preliminary
    plans to convert the \SAIL\ code into a form somewhat like the present
    \quotation {web} were developed by Luis Trabb~Pardo and the author at the
    beginning of 1979, and a complete implementation was created by Ignacio~A.
    Zabala in 1979 and 1980.
\stopquotation

It's here the transition from basic to plain happened. The old book discusses the
basic format, but the final \TEX\ reference talks plain.

\startquotation
    The \TeX82 program, which was written by the author during the latter part of
    1981 and the early part of 1982, also incorporates ideas from the 1979
    implementation of \TeX\ in \MESA\ that was written by Leonidas Guibas, Robert
    Sedgewick, and Douglas Wyatt at the Xerox Palo Alto Research Center. Several
    hundred refinements were introduced into \TeX82 based on the experiences
    gained with the original implementations, so that essentially every part of
    the system has been substantially improved. After the appearance of
    \quotation {Version 0} in September 1982, this program benefited greatly from
    the comments of many other people, notably David~R. Fuchs and Howard~W.
    Trickey. A final revision in September 1989 extended the input character set
    to eight-bit codes and introduced the ability to hyphenate words from
    different languages, based on some ideas of Michael~J. Ferguson.
\stopquotation

By 1989 the program was pretty stable, and we see little changes in for instance
plain \TEX\ in the archives. When listening to the talks, especially from session
9 onward, where files are discussed, you will notice that on the one hand there
is this 127 characters \quote {limitation} imposed, very likely a side effect of
the Computer Modern fonts having at most that many glyphs, although on the other
hand the \TFM\ and \DVI\ formats are capable of more. So, going 256 was no big
deal here, and it's the European user groups that took the opportunity to use the
extended repertoire for enhanced versions of the fonts, which in turn resulted in
different encoding schemes, that itself in turn had consequences for e.g.\
hyphenation patterns.

Another interesting remark in the book is the following.

\startquotation
    When a space comes after a control sequence (of either kind), it is ignored
    by \TEX; i.e., it is not considered to be a \quote {real} space belonging to
    the manuscript being typeset. Thus, the example above could have been typed
    as \type {George P\' olya and Gabor Szeg\" o.} \TeX\ will treat both examples
    the same way; it always discards spaces after control sequences.
\stopquotation

This is something that changed over time, these hundreds of refinements, which
shows that this multi-year project was driven by perfection, user feedback and
practical thinking. A simple example shows what happens today:

\startbuffer
\def\`{!} \` \` \`
\def\f{!} \f \f \f
\stopbuffer

\typebuffer

It will give you: \start \pushoverloadmode \inlinebuffer \stop, so the
explanation in the book (and presentations) of how spaces are handled when a
control sequence is read nowadays adapts to the assigned catcodes. You have to
keep in mind that a control sequence normally was only made from letters, the
single character ones that use a non-letter were primitives.

This quote definitely demonstrates how the constraints of those days were dealt
with, and just as watching an old SciFi movie with tubes as displays makes one
wonder why if flat panels were not predicted, it's clear that foreseeing
computers to become 1000 or more times faster and memory going gigabytes was not
easy.

\startquotation
    When you do use the \type {\:} instruction to change fonts, here are the
    rules you need to know. \TEX\ can handle up to 32 different fonts in any
    particular job (counting different sizes of the same style). These 32 fonts
    are distinguished by the least significant five bits of the 7-bit ascii
    character code you type following \quotation {\type {\:}}; if you don't
    understand what this means, use the following code names for your fonts: {\em
    table omitted}. You never refer to a font by its number, always by its code.
    Code \type {A} is treated the same as \type {a}, etc.; but a wise typist will
    consistently use the same codes in any particular manuscript, because later
    \TEX s may allow more than 32 fonts.
\stopquotation

Indeed, today's \TEX\ implementations go way over 32 fonts, use multi-byte names,
ship macro packages that provide complex font setups, and what more. This of course
already had changed in the final version of \TEX.

\startquotation
    Groups within groups will happen only in rather complicated situations, but
    in such cases it is extremely important that you don't leave out a {\tttf
    \letterleftbrace} or {\tttf \letterrightbrace} lest \TEX\ gets hopelessly
    confused.
\stopquotation

It is these (dangerous bend) comments that makes reading this old document
worthwhile! Although nesting groups 300 deep is not something that happens often,
but processing this document in \CONTEXT\ brings us up to 11 levels, while for
instance the \LUAMETATEX\ manual tops at 25. Maybe in the end macro packages
indeed became too complex.

The way a horizontal list becomes a paragraph of lines remained conceptually
mostly the same. The formulas changed a bit and some more features were added. In
a way this is also how macro packages later evolved: driven by demand. In one
presentation, when explaining some exercises and answering questions, Knuth
mentions that solutions to problems can interfere with each other in a larger
setup and indeed this is where writing a large macro package puts some stress on
the authors: integration. The par builder is an example of where (maybe
conflicting) demands meet as we found out when we extended the builder in
\LUAMETATEX. We felt better after hearing him express in one talk that some parts
of the par builder took some iterations to become perfect and that the first
prototype version had a (what could be considered) a bug that no one had noticed
as on the average the results were fine (so today one would just say it was a
feature later to be improved). It's no surprise that when we extended the code we
had to do some hard thinking. There's a reason why the original (reference)
source has some warnings against changes.

The first version of \TEX\ already had the basic algorithms but more was hard
coded. There is a \typ {\tolerance} like parameter but as far as I can see no
\typ {\pretolerance} and \typ {\emergencystretch} but there are a few integer
parameters that drive the process. Justification is basically limited to flush
right with a parameter that determines the raggedness. There are no \typ
{\leftskip} and \typ {\rightskip} yet and likely for a good reason: typesetting
books is the target. He demonstrates that by messing around with glue in special
ways one can achieve left and right ragged effects. In one talk Knuth answers a
question about stretch between characters in a word with the remark that this is
not what typographers want. If you look at \PDFTEX\ or macro packages on top of
\LUATEX\ you might encounter additional inter-character kerning, something that
sometimes makes sense in titling, but so far the urge to add inter-character glue
had been suppressed. \footnote {Implementing this is actually rather easy but as
it won't get used it only adds overhead and we don't feel the need to prove that
it can be done.}

An interesting difference with final \TEX\ is that spaces after the right brace
that ends the body of a definition (\type {\def}) are ignored. That for sure
got rid of what we call spurious spaces but the final \TEX\ is more consistent
in no longer doing that. I think that dealing with spaces and line endings has
always been somewhat tricky so it is no wonder this evolved.

There are of course dimensions, and they have the usual units. However, there is
no \type {ex} yet, nor \type {sp}. The mentioned repertoire is: \type {pt}, \type
{pc}, \type {in}, \type {cm}, \type {mm}, \type {dd} and \type {em}. Do you see
what more is missing? \footnote {There are in places references to \type {sp} so
it must have been there at some point.} At the start there were no dimension
registers, only some counters, but when registers showed up it also came with
some changes in primitives that expect dimensions.

Space factors are discussed in a way that suggests that they are hard coded which
is likely due to the fact that the focus is on Latin scripts with well known
punctuation code points. Later that became configurable but the approach remained
the same.

In fact quite a bit is still hard coded like the way hyphenation hooks into the
par builder. However, in appendix H the book suggests that there are built-in
rules while the \WEB\ file shows the ability to load pattern files efficiently.
So here we are in some middle ground between old and new I guess; both identify
themselves in the web files as \TEX 82 anyway. \footnote {When you look at (or
listen to) the way hyphenation patterns are made and applied, and when you
realize that we're at the Artificial Intelligence lab, you can actually consider
this to be an an example of machine learning. After all we have lookups driven by
weights stored in a compact form.}

A substantial bit of the book and talks is about error message and interacting
with the system, which is understandable given the systems used at that time. We
have \CONTEXT\ configured to just quit at an error, report the location, and
users can then fix the issue and run again. Hitting a button in the editor
triggering a few seconds run makes more sense. But I do remember the times when it
was better to note down the issue and hit return to see if \TEX\ could catch up.
\footnote {In this perspective I like to notice that when we moved from \PDFTEX\
to \LUATEX\ (and later \LUAMETATEX) processing the \METAFUN\ manual with
thousands of graphics, color, etc.\ went down from 15 minutes to below 20
seconds. Compiling a 250 page book from \XML\ input (also combined with multiple
\METAPOST\ graphics per page) that could take 4 runs and accumulated to 45
minutes, went below a minute. Typesetting a 300 page complex educational math
book with some three thousand formulas takes 6 seconds on a 2025 Chromebook.
Nowadays processing the \TEX\ book takes less than a second so imagine what it
took in those days. It anyway demonstrates that the efficient original
implementation still pays off.}

Actually, one nowadays needs ways to inform the user about issues because reading
the \quote {transcript} is often not done. In this context it was interesting to
hear on one presentation Don mentioning that a \type {.tex} file produces a \type
{.dvi} and \type {.err} file. I made a mental note to mention this \quote
{interesting} and somewhat depressing suffix here, but then was amused by the
fact that in a next presentation that had become \type {.log} because of user
feedback. Today the problem is that the (progress, diagnostic, warning) messages
scroll by so fast that users probably don't notice them. Do they go to the log
file? \footnote {It is why in \CONTEXT\ we explore the possibilities to visual
feedback via a connected device.}

I tried to locate where the single character primitives were defined and ended up
at the interesting section that describes the magic look up trickery of
inter-math element spacing. As the comments mention \type {\thinmskip} (later to
become \type {\thinmuskip}) and such, it looks like the shortcuts are already
gone, although \type {\quad} is still hard coded. So that manual predates the web
version that has the comment \quotation {\TEX\ version 0.25 as it existed when I
gave twelve lectures on the internal details of \TEX82 in July 1982}, the videos
that is. Although \quote {basic} still fits the talks, and that is what Knuth
refers to, progress was made real time in 1982!

% \startquotation
%     The inter-element spacing in math formulas depends on a $8 \times 8$ table
%     that \TEX\ preloads as a 64-digit string. The elements of this string have
%     the following significance:
%
%     \starttabulate[|||]
%     \NC 0 \NC means no space                   \NC \NR
%     \NC 1 \NC means a conditional thin space   \NC \NR
%     \NC 2 \NC means a thin space               \NC \NR
%     \NC 3 \NC means a conditional medium space \NC \NR
%     \NC 4 \NC means a thick space              \NC \NR
%     \NC * \NC means an impossible case         \NC \NR
%     \stoptabulate
%
%     This is all pretty cryptic, but the \TEX\ manual explains what is supposed to
%     happen, and the string makes it happen.
%
%     A global variable \type {magic_offset} is computed so that if \type {a} and
%     \type {b} are in the range \type {ord_noad .. inner_noad}, then \type
%     {str_pool [ a * 8 + b + magic_offset ]} is the digit for spacing between noad
%     types \type {a} and \type {b}.
%
%     If \PASCAL\ had provided a good way to preload constant arrays, this part of
%     the program would not have been so strange.
%
%     \type {"0234000022*4000033**3**344*0400400*000000234000011*4111102340000"}
% \stopquotation
%
% Then there is some code explained, like \typ {magic_offset = str_start
% [math_spacing] - 9 * ord_noad} that is used for a lookup, in the (a bit
% reformatted \PASCAL\ nostalgia for me) code snippet:
%
% \starttyping
% if r_type>0 then {not the first noad} begin
%     case str_pool[r_type * 8 + t + magic_offset] of
%         "0":
%             x = 0;
%         "1":
%             if cur_style < script_style then
%                 x = thin_mskip_code
%             else
%                 x=0;
%         "2":
%             x = thin_mskip_code;
%         "3":
%             if cur_style < script_style then
%                 x = med_mskip_code
%             else
%                 x = 0;
%         "4":
%             x = thick_mskip_code;
%         othercases
%             confusion("mlist4")
%     endcases;
%     if x <> 0 then begin
%         y = math_glue(glue_par(x),cur_mu);
%         z = new_glue(y);
%         glue_ref_count(y) = null;
%         link(p) = z;
%         p = z;
%         subtype(z) = x + 1;
%     end;
% end
% \stoptyping
%
% There is no need to understand this magic but see how we lookup something
% in that string.

\stopsection

\startsection[title=The basic format]

One thing that you will notice, also in the presentations, is that in addition to
\ASCII, which is mentioned explicitly, Don loved to use some special symbols,
like \type {←} for an assignment instead of an \type {=}. In plain \TEX\ that is
gone although the file starts with some mappings of Don's favorite keyboard,
using the pre-space range of characters. Watch the use of octal here:

\starttyping
\chcode'173←1 \chcode'176←2 \chcode'44←3 \chcode'26←4
\chcode'45←5 \chcode'43←6 \chcode'136←7 \chcode 1←8
\stoptyping

The comment sign is the same so we need a macro to typeset it:

\starttyping
\def\%{\char'45 } % Note, the space after 45 is needed! (e.g.\%0)
\stoptyping

Commands often have short names although nowadays one will not see these:

\starttyping
\def\lft#1{#1\hfill}
\def\ctr#1{\hfill#1\hfill}
\def\rt#1{\hfill#1}
\stoptyping

Here we see a feature not present in \TEX, the size keyword, which stands for the
current horizontal size. Another interesting observation is the extreme value of
\type {1000cm}, something not possible today. But at least it's metric.

\starttyping
\def\rjustline#1{\hbox to size{ % newline therefore space
    \hskip0pt plus1000cm minus1000cm #1}}
\def\ctrline#1<\hbox to size{\hskip0pt plus1000cm minus100Ocm
    #1\hskip0pt plus100Ocm minus100Ocm}}
\stoptyping

There is not yet a repertoire of \type {\tracing...} parameters but a single
reference to an internal parameter zero. In one talk you see an (octal) value
assigned to it, we're talking bits and bytes here. The first integer variable
reflects the tolerance and number eight does something with raggedness. An
example in the book uses a value of 1000 as example but we can't test what that
does.

\starttyping
\def\trace{\chpar0←} \def\jpar{\chpar1←} \def\ragged{\chpar8←}
\stoptyping

although we won't go into much detail, it is nice to know how it all started out.
The \quote {tolerance} is set by a multiplier:

\startquotation
    The number $200$ used to determine feasibility can be changed to $100 n$ for
    any integer $n > 1$ by typing \quotation {\type {\jpar <number>}}, where $n$
    is the specified number. A large value of $n$ will cause \TEX\ to run more
    slowly, but it makes more line breaks feasible in cases where lines are so
    narrow that $n = 2$ finds no solutions.
\stopquotation

Watch how we can make a paragraph more ragged by changing a parameter:

\startquotation
    The instruction \type {\ragged <number>} specifies a degree of \quotation
    {raggedness} for the right-hand margins. If this number is $r$, the line width
    changes towards its natural width by the ratio $r/(100 - r)$. Thus, \type
    {\ragged 0} (the normal setting) gives no raggedness; \type {\ragged 100}
    causes the width of each line to be midway between \type {\hsize} and its
    natural width; and \typ {\ragged 1000000} almost completely suppresses any
    stretching or shrinking of the glue. Some people like to use this \quotation
    {ragged right margin} feature in order to make the output look less formal,
    as if it hadn't actually been typeset by an inhuman computer. (Some people
    also think that \quotation {ragged right} typesetting saves money. On
    traditional typesetting equipment, this was true, but computer typesetting
    has changed the situation completely: the most expensive part of the
    computation is now the breaking of lines, while the setting of glue costs
    almost nothing.)
\stopquotation

Currently this is done by a combination of tolerance and left and/or right skips,
in upto three passes (in \LUAMETATEX\ we can specify a sequence of passes so that
can be seen as a \quote {many years later} follow up then).

Of course there are some math definitions, watch the way characters are defined:
by value. The \type {\limitsswitch} is what later became \type {\limits}.

\starttyping
\def\log{\mathop{\char'154\char'157\char'147}\limitswitch}
\def\lg{\mathop{\char'154\char'147}\limitswitch}
\def\ln{\mathop{\char'154\char'156}\limitswitch}
\def\lim{\mathop{\char'154\char'151\char'155}}
\def\limsup{\mathop{\char'154\char'151\char'155
    \,\char'163\char'165\char'160}}
\def\liminf{\mathop{\char'154\char'151\char'155
    \,\char'151\char'156\char'146}}
\def\sin{\mathop{\char'163\char'151\char'156}\limitswitch}
\def\cos{\mathop{\char'143\char'157\char'163}\limitswitch}
\def\tan{\mathop{\char'164\char-141\char'156}\limitswitch}
\def\cot{\mathop{\char'143\char'157\char'164}\limitswitch}
\def\sec{\mathop{\char'163\char'145\char'143}\limitswitch}
\def\csc{\mathop{\char'143\char'163\char'143}\limitswitch}
\def\max{\mathop{\char'155\char'141\char'170}}
\def\min{\mathop{\char'155\char'151\char'156}}
\def\sup{\mathop{\char'163\char'165\char'160}}
\def\inf{\mathop{\char'151\char'156\char'146}}
\def\det<\mathop{\char'144\char'145\char'164}}
\def\exp{\mathop{\char'145\char'170\char'160}\limitswitch}
\def\Pr{\mathop{\char'120\char'162}}
\def\gcd{\mathop{\char'147\char'143\char'144}}
\def\choose{\comb()}
\def\leftset{\mathopen{\{\,}}
\def\rightset{\mathclose{\,\}}}
\def\modop{\<\,\mathbin{\char'155\char'157\char'144>\penalty900\<\,}
\def\mod#1{\penalty0\;(\char'155\char'157\char'144\,\,#1)}
\def\eqv{\mathrel\char'421 }
\def\neqv{\mathrel{\not\eqv}}
\stoptyping

I think that talking in octal was quite popular in those days, just like
hexadecimal is in our times. In later versions of basic we see this:

\starttyping
\def\sin{\mathop{\char s\char i\char n}\limitswitch}
\stoptyping

In plain we eventually got to this: \footnote {In \LUAMETATEX\ we have more
built-in classes and one can add even more, so there we use a dedicated function
class so that we can tune spacing better. So there is a different way to define
functions, also because we can set more properties.}

\starttyping
\def\sin{\mathop{\rm sin}\nolimits}
\stoptyping

This definitions calls for \type {\quad} which is not defined, so it's still a
primitive.

\starttyping
\def\qquad{\quad\quad}
\stoptyping

The \type {\≥} is likely some spacing directive and uses a symbol that even today
is not on keyboards:

\starttyping
\def\ldots{{.\≥.\≥.}}
\def\cdots{{\char'401\≥\char'401\≥\char'401}}
\def\ldotss{{.\≥.\≥.\≥}}
\def\cdotss{\cdots\≥}
\def\ldotsm{{\≥.\≥.\≥.\≥}}
\def\vdots{\vbox{\baselineskip 4pt\vskip 6pt
    \hbox{.}\hbox{.}\hbox{.}}}
\stoptyping

These definitions, related to alignments, are not that different from what they
later became. Again we have rather large plus and minus values. The short \type
{\dispstyle} later became more verbose, so it is not a typo. Does anyone today
use these pile commands, a two-line helper (meant for splitting a long math line
in display mode), chops and/or sposes? The \type {\spose} definitely is invalid
on later \TEX\ engines as they have put a limit on the size of dimensions.

The \type {\!} is a hard coded primitive that basically ends a line
without adding a space. To quote the manual:

\startquotation
    First, a (carriage-return) always counts as a space, even when it follows a
    hyphen. If you want to end a line with a (carriage-return) but no space, you
    can do this by typing the control sequence \quotation {\type {\!}} just
    before the (carriage-return).
\stopquotation

In various places the complications of spacing in the input is discussed and even
today this is something to pay attention to. The \type {\!} is one of the
primitives, the future \type {\ignorespaces}.

\starttyping
\def\eqalign#1{\vcenter{\halign{\hfill$\dispstyle{##}$\!
      ⨂$\dispstyle{\null##}$\hfill\cr#1}}}
\def\eqalignno#1{\vbox{\tabskip0pt plus1000pt minus100Opt
    \halign to size{\hfill$\dispstyle{##}$\tabskip 0pt
      ⨂$\dispstyle{\null##}$\hfill
      \tabskip0pt plus1000pt minus100Opt
      ⨂$\hfill##$\tabskip 0pt\cr#1}}}
\def\cpile#1{\vcenter{\halign{$\hfill##\hfill$\cr#1}}}
\def\lpile#1{\vcenter{\halign{$##\hfill$\cr#1}}}
\def\rpile#1{\vcenter{\halign{$\hfill##$\cr#1}}}
\def\null{\hbox{}}
\def\twoline#1#2#3{\halign{\hbox to size{##}\cr$\quad\dispstyle
    {#1}$\hfill\cr\noalign{\penalty1000\vskip#2}
    \hfill$\dispstyle{#3}\quad$\cr}}
\def\chop to#1pt#2{\hbox{\lower#1pt\null\vbox{\hbox{\lower99pt
    \hbox{\raise99pt\hbox{$\dispstyle{#2}$}}}\vskip-99pt}}}
\def\spose#l{\hbox to 0pt{#1\hskip0pt minus10000000pt}}
\stoptyping

We now arrive at fonts. Given memory constraints the number of fonts that can be
used is small. The names start with \type {\:} and the reference is a character.
These are the precursors of computer modern with predefined sizes, no scaling.

\starttyping
\:@←cmathx
\:a←mr10 \:d←mr7 \:f←mr5
\:g←mi10 \:j←mi7 \:l←mi5
\:n←ms10
\:q←mb10
\:u←msy10 \:x←msy7 \:z←msy5
\:?←mti10
\stoptyping

In later versions of the basic file (dating from after the book) we see that
a\type {c} has been being prepended to the font names. The $10 \rightarrow 7
\rightarrow 5$ steps didn't change over time. These are familiar shortcuts and we
kept them in \CONTEXT, although with a different macro body:

\starttyping
\def\rm{\:a} \def\sl{\:n} \def\bf{\:q} \def\it{\:?}
\stoptyping

There are a few names here that were replaced: \type {\topbaseline} became \type
{\topskip} and the three display skips became four more verbose ones.

\starttyping
\parindent 20pt \maxdepth 2pt \topbaseline 10pt
\parskip 0pt plus 1 pt \baselineskip 12pt \lineskip 1pt
\dispskip 12pt plus 3pt minus 9pt
\dispaskip 0pt plus 3pt \dispbskip 7pt plus 3pt minus 4pt
\stoptyping

Here we see a font switch to an extensible font:

\starttyping
\def\biglp{\mathopen{\vcenter{\hbox{\:@\char'0}}}}
\def\bigrp{\mathclose{\vcenter{\hbox{\:@\char'1}}}}
\def\bigglp{\mathopen{\vcenter{\hbox{\:@\char'22}}}}
\def\biggrp{\mathclose{\vcenter{\hbox{\:@\char'23}}}}
\def\biggglp{\mathopen{\vcenter<\hbox<\:@\char'40}}}}}
\def\bigggrp{\mathclose{\vcenter{\hbox{\:@\char'41}}}}
\stoptyping

Without reading documentation one can guess what this does: we set the
three families:

\starttyping
\mathrm adf \mathit gj1 \mathsy uxz \mathex @
\stoptyping

Eventually \TEX\ got the \type {\textfont} (and script) definition primitives
which gave macro packages the opportunity to use these as macro names.

We're nearly done, here is the basic output routine. It assumes a meaningful
\type {\page}. A page gets the top inserts, followed by the skip, then the
content, then the other skip and finally the bottom inserts. Of course that
became different in \TEX86\ when we got a more generalized approach to inserts.

The baseline skip is quite extreme here, as if it's meant for proofing. Also
watch the page counter: the \type {\count} command is not what we're accustomed
to: it expands, like \type {\the \count} and \type {\number \count} do today.
Some videos refer to \type {\the} so when you watch them and hear a reference to
\quote {sail} or an older \TEX, we're talking this one, otherwise a successor.
The \type {\setcount} does the assignment.

\starttyping
\output{\baselineskip20pt\page\ctrline'{\:a\count0}\advcount0}

\setcount0 1
\stoptyping

Well, why not end the basic file with:

\starttyping
\rm
\null\vskip-12pt % allow glue at top of first page
\stoptyping

This format is small and for a real book more is needed. The assumption was that
there is only a very basic setup with then an additional book specific style. One
could also copy basic and use a patched version. But because those involved were
programmers, or at least aware of what computers could mean for them, it is no
surprise that larger macro packages started showing up. When looking in more
detail at what is provided, in the next sections we will see that some parameters
that went away, others came along, and a couple of concepts changed. That made
writing more general purpose macro packages easier.

\stopsection

\startsection[title=Sail]

The first version was written in \SAIL, and when we fetch the sources from Don
Knuths website we can identify what become primitives and how they are grouped.
Contrary to the next sections, where we can parse the \type {.web} files for
\type {primitive} and use that as trigger for (runtime) filtering, here we show
some more manually filtered sections. If you're not familiar with how \TEX\ is
coded, you have to take what is said below for granted and just try to get the
idea. Seeing the variables and definitions involved can give an impression of how
it all relates to for instance primitives and functionality that you run into.

In \TEX\ we need to distinguish primitives and the way that is done is by
relating them with two numbers: \type {cmd} and \type {chr}. The first 12
commands are single character commands. For instance a backslash \type {chr} is
an escape \type {cmd}. When \type {\something} is encountered in the source,
normally the backslash will trigger reading letters and when done lookup the name
assembled from them. Unless one made an error, that name (it could be a primitive
or user defined macro) is looked up and resolved to a \type {cmd} and \type {chr}
where the latter can be a number representing a value or a pointer to a token
list. We will not go into details here.

An interesting revelation in the presentations is that when you start up \TEX\ in
interaction mode, and no escape character has been setup, the first character
entered will be defined as such, something that of course gets unnoticed when you
start with a command, because then the backslash becomes the escape character.

\starttyping
escape      0 # escape delimiter (\ in TEX manual);
lbrace      1 # begin block symbol ({);
rbrace      2 # end block symbol (});
mathbr      3 # math break ($);
tabmrk      4 # tab mark (ⓧ);
carret      5 # carriage return and comment mark (%);
comment carret is also used as the command code for \cr;
macprm      6 # macro parameter (#);
supmrk      7 # superscript (^);
submrk      8 # subscript (↓);
ignore      9 # chars to ignore;
spacer     10 # chars treated as blank space;
letter     11 # chars treated as letters;
otherchar  12 # none of the above character types;
\stoptyping

These were basically what later became catcode categories, here we have 12, a
modern \TEX\ has 16. It might be a bit out of place, but here is how \CONTEXT\
defines these constants in \MKIX\ using \LUAMETATEX: \footnote {For the record:
the prefixes freezes these definitions (permanent can be overruled by changing an
overload protection flag, the immutable property inhibits even that.)}

\starttyping
\permanent\immutable\integerdef\escapecatcode       0
\permanent\immutable\integerdef\begingroupcatcode   1
\permanent\immutable\integerdef\endgroupcatcode     2
\permanent\immutable\integerdef\mathshiftcatcode    3
\permanent\immutable\integerdef\alignmentcatcode    4
\permanent\immutable\integerdef\endoflinecatcode    5
\permanent\immutable\integerdef\parametercatcode    6
\permanent\immutable\integerdef\superscriptcatcode  7
\permanent\immutable\integerdef\subscriptcatcode    8
\permanent\immutable\integerdef\ignorecatcode       9
\permanent\immutable\integerdef\spacecatcode       10
\permanent\immutable\integerdef\lettercatcode      11
\permanent\immutable\integerdef\othercatcode       12
\permanent\immutable\integerdef\activecatcode      13
\permanent\immutable\integerdef\commentcatcode     14
\permanent\immutable\integerdef\invalidcatcode     15
\stoptyping

Watch the change in terminology: \emphasize {block symbol} is now \emphasize
{group}, \emphasize {math break} became \emphasize {math shift}, and some other
minor renames. But we also got active characters, a configurable comment and, just
to complete the hexadecimal range, an \quote {invalid} category code.

We will go over the rest of the range and comment when we see something
interesting and in most cases the comment explains what we encounter.

\starttyping
parend      13 # end of paragraph;
match       14 # macro parameter matching;
outpar  ignore # output a macro parameter;
endv        15 # end of vlist in halign or valign template;
call        16 # call a user-defined macro;
\stoptyping

In the presentations Don frequently refers to extensions. In current \TEX\
examples of extensions are opening and closing files, reading and writing to
them, and specials. They are nowadays called whatsits but were always part of the
game and lucky us: they made \TEX\ survive decades to come because it could
adapt. For instance \PDFTEX\ uses extensions a lot, for instance for hyperlinks.
\footnote {The original extensions didn't really assume that for instance whatsit
nodes have dimensions, because if that were the case one has to also patch the
various places where dimensions kick in, like in the par builder. In \PDFTEX\
(and therefore \LUATEX) some whatsits do have dimensions so there indeed some
mechanisms have to be aware of that. In \LUAMETATEX\ we treat whatsits as
invisible because, after all, one can wrap into a box to communicate dimensions.}

\starttyping
xt  17 # extensions to basic TEX (\x)
\stoptyping

Here we see dedicated assignment classes for glue and (indeed) reals. In later
versions we talk about (16.16) dimensions and only use \quote {real} for
a field in the box that stored the to be applied glue factor: the only
officially system specific difference across platforms.

\starttyping
assignglue  18 # user-defined glue
font        19 # user-defined current font
assignreal  20 # user-defined length
\stoptyping

We see that the output routine is an independent command:

\starttyping
def     21 # macro definition (\def,\gdef)
output  22 # output routine definition (\output)
innput  23 # required input file (\input)
\stoptyping

There are all kind of parameters but less than we will later get. They don't have
primitive names, and in the basic format we saw \type {\jpar} and \type {\trace}
being defined as shortcuts for numeric references:

\starttyping
setpar  24 # set TEX control parameter (\trace,\jpar)
\stoptyping

In the presentations we also see some tracing that later changed, and debugging
is mentioned too, although it's seen as an emergency measure.

\starttyping
stop  25 # end of input (\end)
ddt   26 # emergency debugging (\ddt)
\stoptyping

This \type {chartype} later became \type {catcode} but what is this \type
{ascii}?

\starttyping
ascii   27 # code for possibly untypeable character (\char)
chcode  28 # change chartype table (\chcode)
\stoptyping

The \type {\char} primitives behave different in text and math mode and later we
got a split between \type {\char} and \type {\mathchar}. The concept of families
is there:

\starttyping
fntfam  29 # declare font family (\mathrm,etc.)
\stoptyping

Here we see a difference with later versions. We have an explicit set and advance
primitives and a serializer. In modern \TEX\ \type {\count} got a different
meaning (setter and getter in one) and \type {\advance} got shared.

\starttyping
setcount  30 # set current page number (\setcount)
advcount  31 # increase current page number (\advcount)
count     32 # insert current page number (\count)
\stoptyping

We're sure that there have been good reasons to change \type {\ifeven} into \type
{\ifodd} even when that introduced an incompatibility, but I guess that given the
pace of development, that was the least of ones worries. The name \quote
{delimiter} for \type {\else} makes me smile. Watch how we have two conditional
command classes. There are no \type {\iftrue} and \type {\iffalse} so here is a
state check example given:

\starttyping
\def\firsttime{T}
... \if T \firsttime{\gdef\firstime{F}}\else{...}\fi ...
\stoptyping

The braces are mandate and also prevent look-ahead expansion as they end the
test. I couldn't find \type {\ifT}, sorry.

\starttyping
ifeven    33 # conditional on count even (\ifeven)
ifT       34 # conditional on character T (\ifT)
elsecode  35 # delimiter for conditionals (\else)
\stoptyping

We still have a split between setter and getter but instead of \type {\save} we
now use \type {\setbox}; we also can make copies. It is worth noticing that we
have a \type {\shipout} called \type {\page} and dedicates justification commands
(\type {\hbox} and \type {\vbox}):

\starttyping
box    36 # saved box (\box,\page) or justification (\hjust,\vjust)
hmove  37 # horizontal motion of box (\moveleft,\moveright)
vmove  38 # vertical motion of box (\raise,\lower)
save   39 # save a box (\save)
\stoptyping

So these were always available:

\starttyping
leaders  40 # define leaders (\leaders)
\stoptyping

These are also familiar:

\starttyping
halign   41 # horizontal table alignment (\halign)
valign   42 # vertical table alignment (\valign)
noalign  43 # insertion into halign or valign (\noalign)
vskip    44 # vertical glue (\vskip,\vfill)
hskip    45 # horizontal glue (\hskip,\hfill)
vrule    46 # vertical rule (\vrule)
hrule    47 # horizontal rule (\hrule)
\stoptyping

However, inserts were hard coded. I think that in these days graphics were still
glued in so basically only footnotes were in demand.

\starttyping
topbotins  48 # inserted vlist (\topinsert or \botinsert)
\stoptyping

In the book style example in the basic \TEX book we see this definition:

\starttyping
\def\footnote#1#2{\botinsert{\hrule width5pc \vskip3pt
  \baselineskip9pt\hbox par size{\eightpoint#1#2}}}
\stoptyping

So we use the bottom insert, which eventually will be separated by a \type
{\botskip} and it will have a rule on top (question: what will happen with
multiple notes? We didn't mention specifying boxes yet but here is the possible
syntax:

\starttyping
\hbox {} \hbox to size {} \hbox to <dimen> {} \hbox expand <dimen> {}
\hbox par size {} \hbox par <dimen> {}
\vbox {} \vbox to size {} \vbox to <dimen> {} \vbox expand <dimen> {}
\stoptyping

This \type {par} is used in the footnote definition, we basically get a \type
{\vbox} disguised as \type {\hbox}. Later these box definitions became:

\starttyping
\hbox {} \hbox to <dimen> {} \hbox spread <dimen> {}
\vbox {} \vbox to <dimen> {} \vbox spread <dimen> {}
\stoptyping

At that point we also got \type {\vsplit}, multiple inserts, inserts that could
be split over pages, properties like skips bound to inserts, etc.

The, also migrating, marks look familiar and two command categories are used.
Here \type {\firstmark} is not mentioned:

\starttyping
topbotmark  49 # insert mark (\topmark,\botmark)
mark        50 # define a mark (\mark)
\stoptyping

For sure we have:

\starttyping
penalty  51 # specify badness of break (\penalty)
\stoptyping

And:

\starttyping
noindent  52 # begin nonindented paragraph (\noindent)
\stoptyping

Eh \unknown\ we now have triggers like \type {\outputpenalty} and the concept of
push back in the output routine. The old \TEX\ engine has an explicit \type
{\eject} that can be invoked in horizontal mode (middle of a paragraph) or
vertical mode. In new \TEX\ this became a plain macro. Another interesting
primitive is \type {\page} that inserts the collected content, box number 255 in
new \TEX.

\starttyping
eject  53 # eject page here (\eject)
\stoptyping

The next command deals with the hyphen related primitives. The \type {\*}
primitive is for math where it will insert a \im {\times} with a possible line break
after it. This is an interesting observation: this native feature was removed in
the final version of \TEX, but at some point in \LUAMETATEX\ discretionaries came
back, for repeated operators (at the end of a line and the start of a next line)
as well as specific (three part) breakpoints.

\starttyping
discr  54 # discretionary hyphen (\-,\*)
\stoptyping

Interesting, as later we only had an accent placement, seldom seen in the running
text, possibly used in macros, if at all, because at some point \TEX\ went eight
bit.

\starttyping
accent     55 # attach accent to character (\+)
newaccent  56 # define nonstandard accent (\accent)
\stoptyping

Only one side of the (centered) equation is covered:

\starttyping
eqno  57 # insert equation number (\eqno)
\stoptyping

Hm, apart from that space, we don't have it like this any longer:

\starttyping
mathonly       58 # character or token allowed in mathmode only
exspace        59 # explicit space (\ )
nonmathletter  60 # letter except in mmode
\stoptyping

In modern \TEX\ these became fences:

\starttyping
leftright  61 # variable delimiter (\left, \right)
              # comment there is no code 62 today
\stoptyping

These are atoms (a nucleus with a superscript or subscript):

\starttyping
mathinput  63 # component of math formula (\mathop,\mathbin, etc.)
\stoptyping

We still have this modifier, a primitive that adapts the last added math atom (if
it makes sense): \footnote {The node list is forward linked only in traditional
\TEX\ so this only works on what is called the tail of the current (math) list. In
\LUATEX\ we are dual linked so in principle one can look further back but there
is no real reason to do this. It is one of the few cases where the engine looks
back anyway.}

\starttyping
limsw  64 # modify limit conventions (\limitswitch)
\stoptyping

Math again, but the \type {\comb} is gone. We have the \type {\..withdelim}
variants now:

\starttyping
above  65 # numerator-denominator separator(\above,\atop,\over,\comb)
\stoptyping

Watch how for instance the text style is referred as:

\starttyping
mathstyle  66 # style or space specification (\dispstyle,\,,etc.)
\stoptyping

We still have that, it's a typical \TEX\ concept, not present in modern fonts:

\starttyping
italcorr  67 # italic correction (\/)
\stoptyping

Yes, another adjust. In traditional \TEX\ this is a math mode command, but in
\LUAMETATEX\ we made it valid in text mode too.

\starttyping
vcenter  68 # vjust centered on axis (\vcenter)
\stoptyping

This next command is peculiar because it is not a parameter as the other integer
ones but it has it's own command category. There is in this list of commands no
mentioning of more advanced paragraph properties, not even \type {\everypar}
because that will show up in \TEX82.

\starttyping
hangindent  69 # specifies hanging indentation (\hangindent)
\stoptyping

So where is \type {\hangafter}? It actually is the reason why we have a dedicated
command class because a special scanner is needed:

\starttyping
\hangafter 20pt for   1
\hangafter 20pt after 2
\stoptyping

A negative dimension hangs right, the \type {for} and \type {after} options
determine the number of lines.

Next we see some of the parameters mapped to memory. We challenge you to translate
these to modern \TEX's parameters:

\starttyping
define tracing = eqtb[hashsize+268] # controls diagnostics
define jpar    = eqtb[hashsize+269] # controls justification
define hpen    = eqtb[hashsize+270] # hyphenation penalty
define penpen  = eqtb[hashsize+271] # penultimate penalty
define wpen    = eqtb[hashsize+272] # widow-line penalty
define bpen    = eqtb[hashsize+273] # broken-line penalty
define mbpen   = eqtb[hashsize+274] # binary-op-break penalty
define mrpen   = eqtb[hashsize+275] # relation-break penalty
define ragged  = eqtb[hashsize+276] # raggedness
\stoptyping

The par builder can be configured a bit but some criteria are hard coded. Watch
how we don't have a club penalty here. The widow penalty is used for that as well
as display math.

Because skips are more than just a single integer, they are actually nodes, so
here we need to refer to node memory:

\starttyping
define lineskiploc     = locs[0]
define baselineskiploc = locs[1]
define parskiploc      = locs[2]
define dispskiploc     = locs[3]
define topskiploc      = locs[4]
define botskiploc      = locs[5]
define tabskiploc      = locs[6]
define dispaskiploc    = locs[7]
define dispbskiploc    = locs[8]
\stoptyping

Just to show you that the math concepts are there:

\starttyping
define dispstyle         = 0
define textstyle         = 1
define scriptstyle       = 2
define scriptscriptstyle = 3
\stoptyping

Nowadays we still have these noads, intermediate nodes, here are the atoms:

\starttyping
define boxnoad   = 0
define opnoad    = 1
define binnoad   = 2
define relnoad   = 3
define opennoad  = 4
define closenoad = 5
define punctnoad = 6
\stoptyping

Do you recognize the fraction?

\starttyping
define sqrtnoad   =  7
define overnoad   =  8
define undernoad  =  9
define accentnoad = 10
define abovenoad  = 11
\stoptyping

The square root has no degree but other roots have. Eventually \UNICODE\ engines
using \OPENTYPE\ math fonts got a variant that handles the degree option but here
it still has to be done manually via a macro that overlays the degree.

Nowadays we have a fence noad to which \ETEX\ added a middle variant (subtype):

\starttyping
define leftnoad  = 12
define rightnoad = 13
\stoptyping

The simple noad and style are there too but we can assume that the way the noads
map onto memory changed a bit over time (I didn't check it).

\starttyping
define nodenoad  = 14
define stylenoad = 15
\stoptyping

These look more hard coded than in the final version of \TEX, where we can define
and set glue and muglue registers.

\starttyping
define thinspace     =  8
define thickspace    =  9
define quadspace     = 10
define negthinspace  = 11
define negthickspace = 12
define negopspace    = 13
define userspace     = 14
define nospace       =  6
define opspace       =  7
define thspace       = 15
define negthspace    = 16
\stoptyping

This was just an impression and likely one that has errors. We don't have (access
to) many details but even if we're wrong in aspects it is clear using this \SAIL\
prototype gave plenty of input to the project so when the rewrite to \PASCAL\
took place, mixed with \WEB\ documentation, these concepts evolved. Let's now
move to the next iteration but before we do that we zoom in on the transition
between the prototype and what became the final version.

\stopsection

\startsection[title=The intermezzo]

The book has an appendix \type {<X>} titled \quotation {Recent extensions to
\TEX} that describes additions to the engine done right before the mentioned book
was published. These were likely driven by usage, macro packages showing up, and
users demanding more from macros and manipulations. In the videos Knuth makes
clear that \TEX82 will converge to a stable version. He announces the \TEX\ book,
the \METAFONT\ book and a book about the fonts. He also mentions that, while \TEX\
underwent rather fundamental changes and upgrades, \METAFONT\ will be less
different. But when you look at the \METAFONT\ section in the book, it's hard to
see the similarities. Of course I look at it from the perspective of \METAPOST\
being used for graphics other that glyph shapes, but if we look at the bitmap
generation bits, the language, variables, path construction, macros \unknown\ one
really has to rewrite the older code, but most ideas behind it remain. References
to built-in variables, hard coded pens, control point features, subroutines and
calls, it looks somewhat less meta than today.

But when it comes to \TEX, in a way you can recognize the upcoming \TEX 82 in
that because when extensions like that get added, later changes are natural. So,
let's reflect a bit on what we can read about it. Take for instance how we
stepwise came to \type {\advance}. There are ten counters and you can do the
following

\starttyping
\setcount 4 = 10  \advcount 4  \advcount 4  \count 4
\stoptyping

And because \type {\count} expands you get a typeset 12 here because \type
{\advcount} increments by one. In the appendix an extension is introduced:

\starttyping
\setcount 4 = 10  \advcount 4 by 2  \count 4
\stoptyping

This comes in handy when we need to increment by 50 which otherwise would demand
a recursive loop. However, in \TEX82 the \type {\advcount} got companions like
\type {\advdimen} and code was shared between, made possible by (internally)
assigning a property that signals what is dealt with. But, soon an incompatible
change was introduced:

\starttyping
\count 4 = 10  \advance\count 4 by 2  \the\count 4
\stoptyping

From that time on each register uses the same \type {\advance} and the now
unexpandable register references have to use \type {\the} as serializer. This
transition was rather natural and made it easier to program more complex
solutions.

We didn't mention yet that the conditional primitives were also syntactically
different. It started with:

\starttyping
do \ifeven \count4 {this}\else{that}
\stoptyping

so mandate braces, a mandate (redundant?) \type {\else} and no \type {\fi}. That later
became:

\starttyping
do   \ifodd \count4 this\else that\fi % if it has to be clear
do th\ifodd \count4   is\else   at\fi % if you want to impress
\stoptyping

When the appendix introduces the \type {\ifpos} test, the next example is given:

\starttyping
\def\neg#1{\setcount#1-\count#1}
\def\ifzero#1#2\else#3{\ifpos#1{#3}\else{\neg#1
    \ifpos#1{\neg#1 #3}\else{\neg#1 #2}}}
\stoptyping

Apart from \type {\neg} later being a math related command, this looks like a lot
of work for a test on a number being positive. It is not fully expandable (but one
can wonder if that mattered much in those early days), it negates the counter
because it can only test for positive and then has to convert the counter back to
its original value. How much more convenient will this become:

\starttyping
\ifnum#1>0 ... \else ... \fi % replaces \ifpos
\ifnum#1=0 ... \else ... \fi % replaces \ifzero
\ifcase#1  ... \else ... \fi % alternative
\stoptyping

It is revealing to read about the development of (conditionals in) early \TEX, in
particular regarding recent choices I made in \CONTEXT\ when extending the
repertoire. As a \LUAMETATEX\ extension \type {\orelse} is one of my favorites
because it makes for less nesting of conditionals. These types of simple
constructions are the core of the language and of immediate use as soon as a user
needs to go a bit low level. As part of the soul of the \TEX\ language we are
happy to expose our users to them.

Let's just mention that \typ {\ifvmode}, \typ {\ifhmode}, \typ {\ifmmode} were
added as well as \typ {\lineskiplimit} and \typ {\mathsurround} and more would
follow.

Yet another extension holds some promise for the future \TEX82: a user dimensions
(there are no dimen registers yet). The \type {\varunit} variable can be set to
some value and the \type {vu} dimension can then use that. In order to be able to
work with the dimensions of boxes there were also \type {wd}, \type {ht} and
\type {dp} units each to be followed by some box number.

The fact that one could say \type {12.3vu} or \type {2wd3} actually makes me feel
less guilty for adding the \type {dk} unit (as a demonstration of extension), the
\type {ts} and \type {es} units (in order to celebrate the fact that teenagers
are willing to accompany parents to \TEX\ conferences) and \type {eu} for what
actually is a bit like the var unit, in this case a multiplier for \type {es},
and therefore European thumb based (to counter the inch). But aside from these
semi-serious extensions, in \LUAMETATEX\ we also have installable units, so we
could do this if we like to:

\starttyping
\newdimen\varunit \varunit 10pt
\associateunit\varunit{(`v-`a)*26+(`u-`a)}
\varunit 10pt \dimen0=10vu \the\dimen0
\stoptyping

It happens that we already have \type {uu} as \quote {user unit} but if I'd known
that there had been a \type {vu} \unknown\ who knows. We could discuss it at an
upcoming \CONTEXT\ meeting.

But there is another guilt diminishing feature: this box dimension unit. It
demands not just a keyword check but also picking up a number. Later we got \type
{\wd} etc.\ that are a but more natural in accessing these properties, but the
fact that we have a kind of a sub-expression makes that adding plenty more such
specific burdens on the parser seems quite okay. So where in the above snippet
you will notice that we scan an expression, you can be sure that more is possible
between these braces and it's not that unnatural to \TEX\ after all.

In today's \TEX\ there are ways to set up characters to be of a certain math
class. In the intermezzo between \SAIL\ \TEX\ and \TEX82 that could be done via
the what was then the catcode setter: \type {\chcode}, using values beyond the
catcode range. Later math got its own setters for that. At the same time the
\type {\char} primitive got the option to pass a letter as alternative for a
number. That was later changed, because we got \type {`} as numeric prefix to
characters, like \type {`a}. I suppose some users had to change their format
files and styles a few times. It might also have been one of the reasons why
later \TEX\ got frozen, not only because of the rendering but also because of
setting it up to do just that. \footnote {There is this active character class in
math mode so in some sense we still have a special relationship between character
codes and math classes.}

The somewhat weird, large \type {plus} and \type {minus} values in some basic
macros are so large that later they became invalid. In the intermezzo period
we see \type {\hfil}, \type {\hfilneg}, \type {\hss} and their vertical
counterparts replace them: no more \typ {plus 100000pt} and such are needed. The
argument is actually that it saves memory, which is due to the fact that instead
of variables (of glue nodes) one has references to fixed nodes. Of course one can
wonder if in practice it really saved much memory. A similar optimization, namely
sharing space glue by referencing glue nodes fits in here too but isn't mentioned.

The introduction of \type {\xdef} was needed because there was no \type {\global}
prefix yet. Introducing the \type {\lowercase} and \type {\uppercase} primitives
made sense but later usage demonstrated that as they operate on their argument
and not the node list it is not always trivial to use them in situations where
expansion can interfere.

A final remark in the appendix mentions that control sequences are from then on
remembered in full, that is: the seven character limit is gone! It probably also
made for less code. That said, we now move on to the first \TEX82\ implementation
that figures in the presentations.

\stopsection

\startsection[title=Old \TEX82]

When we talk about \quote {old \TEX}, we mean the early '82 version. The sources
can be downloaded from Don Knuths website but they are just that: sources. There
is nothing you can run so your mind is the computer. For that reason we will only
look at the commands, and comment a bit on those. We are of course talking of a
working program but when you compare it with what we will call \quote {new \TEX82}
you will notice some differences that indicate that the latter is more mature
and that decisions were made that really made sense for the program to succeed
and prosper. We start with the primitives and present the list as filtered from
the source(s). Bear in mind that we could have missed something. The idea is that
you go over the list and try to identify commands that are gone or changed. The
first column mentions the primitive, the second column the so called command
category it fits in. You can think a bit of the primitives being interpreted by a
virtual machine that interprets an operator and operand.

It is a long list but we like you to try to locate the primitives that are not in
today's \TEX, that have been renamed, or have become different.

% \definetypeface
%   [narrowtt] [tt]
%   [mono] [modern-condensed] [default] [features=none]
%
% \start \narrowtt \setuptype[style=narrowtt]

\start \small
    \ctxlua{document.showprimitives("old")}
\stop

The list of commands that the primitives are grouped in:

\startcolumns[n=3]
    \ctxlua{document.showcommands("old")}
\stopcolumns

So what are the most significant differences with the \SAIL\ version? We see many
more primitives and quite some are for setting and getting parameters. These are
now organized into integers, dimensions, glue, muglue and token lists.

A noticeable difference with today's \TEX\ is the \type {set_font} that uses this
\type {\:} shortcut. It's good to know that we still have limited memory.
Machines got more powerful but even a DEC 10 (or 20) had to serve more than one
user and given that each seemingly had 500~kB running \TEX\ puts some strain
on the system (swapping memory and such). It is why there were two versions, one
that could dump a format and another that just loaded it and therefore could drop
some code in favor of using more memory for processing. But fonts, although small
by today's standards took space so the number was limited. On the other hand, we
got away from the short (sometimes cryptic) names and have way more primitives.
It feels good.

Watch commands like \type {\texinfo}, \type {\groupend} and \type {\groupbegin},
\type {\hangindent} with its own command class, \type {\xoverx}. But also look at
the various \type {\adv...} commands and \type {\minus}. Conditionals have this
\type {\ifeven} the future negated \type {\ifodd}; the \type {\ifabsent} check
will at some point become \type {\ifvoid} and \type {\case} will migrate to \type
{\ifcase}. Also consider \type {\varunit} and \type {\:} as we will see these
change later.

We see the concept of mu glue show up, and we see \type {\thinmskip} pop up.
Interestingly that one became \type {\thinmuskip}, probably because \type
{\mskip} is a more general skipper and therefore we need to avoid confusion. It's
in the details. Going from \typ {disp} to \type {display} also feels natural. We
see \type {\xabovex} and companions, \type {\chcode}, quite some \type {\set...},
\type {\adv...}, \type {\mult...} and \type {\div...}. Maybe if there had't been
so many we'd have ended up with \type {\add} and \type {\subtract} instead of
just \type {\advance}.

Watch how the \type {\shipout} is treated as a leader. The operands (\type {chr})
now use more symbolic names but not all of them. Interaction internally uses a
variable with four values but each got its primitive. The tracing options were
split into independent parameters.

\stopsection

\startsection[title=New \TEX82\endash89]

The current version of \TEX\ has the next list of primitives. There are some
differences in names but more significant is that we have a different grouping in
commands. As with the previous versions, not all operands have symbolic names.

\start \small
    \ctxlua{document.showprimitives("new")}
\stop

Again we show the list of commands. It's about as large as the previous one
although we got rid of some. Looking at the list is a test of how well you know
the current set of official \TEX\ primitives.

\startcolumns[n=3]
    \start \small
        \ctxlua{document.showcommands("new")}
    \stop
\stopcolumns

We will make it easier to see the changes in organizing the commands. The second
column shows the number of primitives in the older version, the third column
refers to the current version. The bold entries are unchanged names.

\startcolumns[n=2]
    \start \tx
        \ctxlua{document.showcombined()}
    \stop
\stopcolumns

It is interesting to see how \TEX\ evolved and got more organized with each
iteration. It also made me feel a bit less for reorganizing these commands a bit
more. In \LUATEX\ we added primitives, just like for instance \ETEX\ and \PDFTEX\
had done. In \LUAMETATEX\ we added even more. In one of the presentations Don
shows a trick to add two dimensions: do two \type {\hskip}'s in a \type {\hbox}
and use the width as sum. He then explains that one can also subtract, multiply
and divide using such trickery. However, in the final version of \TEX\ we got
\type {\advance}, \type {\divide} and \type {\multiply}, and \ETEX\ later added
simple expressions. So in the end, the fact that \LUAMETATEX\ got a more powerful
expression mechanism and additional data types feels kind of natural. So, why not
show the \LUAMETATEX\ command grouping here? This is what we have 25 years later:

\startcolumns[n=2]
    \start \tx
        \ctxlua{document.showluametatex()}
    \stop
\stopcolumns

Some are renames, others regrouping, but there are also new ones. For instance
the command group \type {association} deals with user defined units (bound to a
register, macro or some \LUA\ call). The \type {begin_local} relates to \type
{\localcontrol} and friends, a way to exercise the main loop inside a macro and
avoid side effects; a bit similar to \type {vardef} in \METAPOST. Boundaries are
new and have a \type {boundary} command group. The various parameters (variables)
are split into internal and register (user) ones.

Not all commands have a primitive (this is also true for the predecessors where
we didn't show them) but here we just show them with no primitive count. We have
registers (way more than original \TEX) but also have additional pseudo registers
that are basically (efficient) macros. This means that we can generate a version
with less registers which, even with a decent set, saves about the same amount of
memory that good old \TEX\ has available for processing documents, including its
own binary. \footnote {The \ETEX\ extension pushed \TEX\ from 256 to 32.786 and
\LUATEX\ doubled that but it makes no sense to have that many: who needs that
many muskips or attributes? Some 8K seems more than enough for each with probably
4K for the glue and muglue.} Anyway, some of the codes with no number have
similar command codes in old and new \TEX, we didn't invent all those wheels.

It makes no sense to show the new primitives but we have for instance iterators
that have a command category but primitives that are in other groups. This has to
do with the way they behave. Of course we have a bunch of \LUA\ related ones too.
The \type {\..._call} commands are fast accessors to specific definitions, like
\type {\protected} or \type {\tolerant} ones. Command groups like \type
{active_char} and \type {letter} are also present in the reference \TEX\ engine,
as they relate to the catcodes, but characters are not primitives. The fact that
we have \UNICODE\ and therefore huge slices of potential entries in \TEX's
equivalents space already makes for different internals anyway. But looking at
the tables of various engines can at least give an idea of how the engines
evolved. It definitely made watching these 40 year old videos so much more
interesting and fun.

\stopsection

\startsection[title=The timeline]

We started with a rough timeline but there is a very good source for detective
work: the \type {errorlog.tex} file also typeset in Knuths Digital Typography. I
wish I had the discipline to come up with something like that but I can use the
fact that I write intermediate wrap-ups as a lame excuse. From 1978 onward you
can read what bugs were solved, what renames happened, what got introduced, etc.
Let's mention a few entries that relate to things discussed earlier in this
document, especially in the intermezzo section.

At 19 May 1978 we note \quotation {G249. Add a \type {\topbaseline} feature
[later called \type {\topskip}]. @1001}. Before that we had some skips after top
inserts and before bottom inserts but these became skip registers eventually. On
5 Aug 1978 we read \quotation {G336. Generalize \type {\pageno} to \type {\count
<digit>}. @236} but \type {\pageno} later became the de-facto standard for \typ
{\the \count 0} so it never really went away. From mid 1979 it becomes clear
that the new par-builder is being worked on, we also see the mentioned fillers
show up on 23 Jul 1979: \quotation {E409. Introduce new primitives \type
{\hfil}, \type {\vfil}, \type {\hfilneg}, \type {\vfilneg}. @1058}

Not mentioned in the book as upcoming are active characters. These are mentioned
25 Jan 1980: \quotation {G427. Introduce active characters (one-stroke control
sequences). [I don't yet go all the way: The meanings of \type {x} and \type {\x}
have to be identical.] @344} Actually, active characters are an interesting concept
although in today's usage (think \UNICODE) they make less sense.

We already saw \type {\def} and variants being in the \SAIL\ version but in 1980
we also got \quotation {G444. Add a new \type {\newname} feature (soon changed to
\type {\let}). @1221}.

A milestone is 13 Jun 1980: \quotation {Today I'm beginning to overhaul the
line-breaking routine, and I'll also install miscellaneous goodies.} There is
also a new feature: \quotation {G459. Add a new parameter \type {\loose} [later
\type {\looseness}]; now parameters are allowed to take negative values. @875}. A
day later 14 Jun 1980 we read \quotation {Q461. Install new line-breaking
routines, including \type {\parshape}. (These major changes are introduced as
Michael Plass and I write our article.) @813}.

The inserts are maturing: \quotation {G482. Add new \type {\topsep} and \type
{\botsep} features. [These are \TEX78's way to put space at the edge of inserts,
replaced in \TEX82 by the \type {\skip} register corresponding to an \type
{\insert} class.] @1009}. These changes happen when Don is also writing the
upcoming books, and I think it demonstrates his valid point that writing a manual
is a great way to improve the code, for instance with respect to consistency.
Implementation, documenting and using go hand-in-hand.

Here's one for us: \quotation{G490. Add the dimension \type {cc} for European
users. @458}. I never used it as I prefer \type {cm}, \type {mm} and \type {pt},
as silent protest against the American \type {in} and \type {bp}.

This is a nice one: \quotation {C491. Make \type {scan_keyword} match uppercase
letters as alternatives to lowercase ones (suggested by Barbara Beeton's
experiments with \type {\uppercase}). @407}.

Then there is the comment \quotation {Am freezing current program as version
$-0.25$; a week of \TUG\ lectures begins tomorrow.} Are these the ones from the
video? Was that something \TUG ? In one of those videos the 5 Aug 1982 change
\quotation {IX158. The \type {.err} file should be \type {.log} instead. @534} is
announced, a decision driven by user input during a break in the course.

Around that time there is also the optimistic \quotation {I believe the \type
{line_break} routine has passed its test perfectly.} and then \quotation {FX247.
Initialize \type {second_indent} in the easy case. @848}. \footnote {Mikael and I
still try to understand what this easy case refers to, other than that it relates
to looseness in the current version. Thanks to \quotation {IX334 X199. Introduce
serial numbers in line-break records, improving readability and independence.
@846} we can indeed see how many passive nodes get set.}

The class related character definitions show up: \quotation {GX260. Introduce new
primitive \type {\mathchardef}, to save space and time. @1224} which is nicer
that a mix with character catcodes. In this perspective it's good to remind the
reader that we have at most 256 characters in a font. In a modern \UNICODE\ font
we have way more which also means that we have character codes that need a 32 bit
integer. Some of the original \TEX\ approaches in \LUATEX\ got a different
implementation, for instance by using sparse arrays. \footnote {In \type
{musing-neat.tex} I explore a bit how characters are stored in a node list, the
suggested way to handle large fonts and why we do it differently.}

Relatively late we see \type {\:} being replaced; the videos indeed still use
that short command: \quotation {G545. Install a major change: Fonts now have
identifiers instead of code letters. Eliminate the \type {\:} primitive, and give
corresponding new features to \type {\the}. @209}.

The par builder gets an update: \quotation {A554. Compute demerits more suitably
by adding a penalty squared, instead of adding penalties before squaring. @859}
and \quotation {Previously a slightly loose hyphenated line followed by a decent
line was considered worse than a decent hyphenated line followed by a quite loose
line.} Also \quotation {E566. Omit the \quote {first pass} and try hyphenations
immediately, if \type {\pretolerance} is negative (suggested by DRF). @863}. How
many users do the latter? We mentioned this already but it happens late 1982:
\quotation {C578. Change \type {\hangindent} to a normal dimension parameter. [It
had been a combination of \type {\hangindent} and \type {\hangafter}, with
special syntax.] @247}.

We also mentioned the \type {\ifeven} to \type {\ifodd} change; the new one is
mentioned: \quotation {S591 564. Make \type {\ifodd 1\else} legal by introducing
\type {if_code}. @489}.

The beginning of 1983 also marks the beginning of new grouping names: \quotation
{C597. Rename \type {\groupbegin} and \type {\groupend} as \type {\begingroup}
and \type {\endgroup}. @265}.

Documenting likely lead to: \quotation {C639 607. Remove the kludgy math codes
introduced earlier; make \type {\fam} a normal integer parameter and allow \type
{\mathcode} to equal $2^{15}$. @1233}. We saw in basic that octal was popular but
\quotation {I641 639. Replace octal output (\type {print_octal}) by hexadecimal
(\type {print_hex}) so that math codes are clearer. @67}.

At 22 Feb 1979 we could read \quotation {G387. Add \type {vu} and \type
{\varunit}. [\TEX82 will eventually allow arbitrary internal dimensions as units
of measure.] @453} and that announcement between square brackets lead to 25 May
1983 mentioning \quotation {G695 X231. Remove \type {dm} and \type {vu}; allow
the more general \type {.5\hsize}}. We also read \quotation {@455. C696. Change
\type {\texinfo <f> <n>} to \type {\fontdimen <n> <f>}. @578}.

We also mentioned the introduction of units to access box dimensions but the 27
May 1983 entry tells \quotation {G703 695. Replace (and generalize) the previous
uses of \type {ht}, \type {wd}, and \type {dp} in dimensions by introducing the
new control sequences \type {\ht}, \type {\wd}, and \type {\dp}. @1247}.

If you're interested how software development went in the years 1978 to 1983 this
error log is a must-read because it actually is a development log. Between the
lines you can read about compilation and compiler issues, access to computers
during day and night, and evolving features. It makes one who is only accustomed
to fast personal computers, screen editors, maybe code specific features in
editors, seemingly unlimited memory and disc space a bit more humble. It also
makes clear that thinking beforehand as well as printing and reading a source was
part of a skill-set, if only because trial-and-error or just trial-after-coding
was less of an option.

\stopsection

\startsection[title=Conclusion]

It was interesting to compare the \SAIL/\TEX82 lectures, and \TEX82 as-of-now
with each other, especially because I spent a few decades with \CONTEXT\ friends
on extending \LUATEX\ and later \LUAMETATEX. I had of course browsed the error
log but checking it again in this perspective is nice. Looking at how it all
evolved also gives some clues why things are as they are now; to some extent it
all makes even more sense: limitations and possibilities. It also matters (for
me) that in the meantime, for quite some years, I have been involved in and have
been playing around with extending (core features of) the engine, also in the
perspective of developing \CONTEXT, which has its demands and stresses the
machinery.

One has to go back in time and watch the development process, the teamwork, and
the inspiring leadership, working towards quality, that made it possible. The
success of \TEX\ and friends very much related to the way Don Knuth interacted
with the community, and the effort he put in all this, often setting aside his
main priority, the art books. The result is still pretty valid, and those who
claim otherwise can take a lesson from that. After all, in software development
illogical arguments, sometimes ridiculing (the past), bragging about trivial
achievements, quick and dirty finalizing, lack of quality control and what more,
are quite common in these commercially driven developments. And we're not even
talking of using generative artificial intelligence to come up with new code,
based on old code, presented as original. But watching these videos confirms why
I like \TEX, and why I got attracted to it, seeing the books, understanding
little without the ability to run the programs, picking up on that, and ending up
in the \CONTEXT\ community. Figuring out solutions using a framework for
typesetting, one that had been designed with a future usage in mind, can be fun.
It's pretty hard to beat!

An important lesson is that coming up with a (original) solution (for your
problem) that is also future proof takes time. One needs to play with it and be
willing to look at it from a different angle. It makes little sense to work in
isolation because constructive user input helps. To that I personally add that if
you don't like some development, just ignore it; there is no need to waste time
on competition and claiming that this is better than that and that everyone
should use whatever you like. Just think of this: \TEX\ has been around for
decades for good reasons.

\blank[2*big]

\startlines
Hans Hagen
December 2025
Hasselt NL
\stoplines

\stopsection

\stoptitle

\stopdocument
